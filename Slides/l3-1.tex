\documentclass[professionalfonts]{beamer}
\newif\ifita
%\itatrue % comment out to hide answers
\itafalse
\usepackage[familydefault,light]{Chivo} 
\usepackage[T1]{fontenc}
\usenavigationsymbolstemplate{}
\usepackage[]{hyperref}
\usepackage{tikz,pgf,pgfarrows,pgfnodes,pgfbaseimage}
\graphicspath{{./Pics/}}
\usetikzlibrary{shapes}
\usepackage{setspace}
\newcommand{\evi}[1]{{\colorbox{yellow!50}{{#1}}}}
\newcommand{\exe}[1]{{\color{black!50}{{#1}}}}
\newcommand{\kw}[1]{{\colorbox{black!30}{\color{white}{#1}}}}
\tikzstyle{nd}=[circle,draw=black,thick,minimum size=.8cm,inner sep=1pt]
\setbeamercovered{transparent}
\usetheme{Singapore}
\tikzstyle{nodo}=[ellipse,draw=black!60,fill=black!10,line width=.7pt,minimum width=.7cm,minimum height=.4cm]
\usecolortheme[named=gray]{structure}
\setbeamercolor{block title}{bg=black!20,fg=black}
\setbeamercolor{block body}{bg=black!10,fg=black}

%%%%%%%%%%%%%%%%%%%%%
\ifita
\title{Algoritmi Numerici (Parte III)}
\subtitle{[Lezione 0] Basi di Calcolo Matriciale}
\else
\title{Numerics (Part III)}
\subtitle{[Lecture 0] Linear Algebra Basics}
\fi
\date{}
\author{Alessandro Antonucci\\{\tt alessandro.antonucci@supsi.ch}}
\begin{document}
\maketitle
\setstretch{1.4}
%%%%%%%%%%%%%%%%%%%%%%%%%%%%
\frame{\frametitle{\ifita Vettori \else Vectors \fi}
\begin{itemize}
\ifita
\item Vettore (array): una sequenza ordinata di numeri (reali)
\item Dimensione del vettore: numero di elementi nella sequenza 
\item Distinguiamo vettori riga (``orizzontali'') \\ e vettori colonna (''verticali'')
\item Trasposizione: trasforma una riga in una colonna e v.v.
\else
\item Vector (array): ordered sequence of (real) numbers
\item Vector size: number of elements in the sequence
\item We distinguish ``row'' vectors (horizontal) \\ from ``col'' vectors (vertical)
\item Transpose: swap rows and cols (and vice versa)
\fi
\end{itemize}
\vskip 4mm
$$\vec{x}=[1 \quad 4 \quad 0] \quad \vec{x}^t = \left[ \begin{array}{c} 1 \\ 4 \\ 0 \end{array} \right]$$
$$len(\vec{x})=3 \quad (\vec{x})_1 = 1 \quad (\vec{x})_2 = 4$$}

\frame{\frametitle{\ifita Calcolo Vettoriale \else Vector Calculus \fi}
\begin{itemize}
\item \ifita Scalare: ``array con un solo elemento'' \else Scalar: single-element array \fi
\item \ifita Moltiplicare uno scalare per un array \`e operazione \emph{elementwise} \else
Multiplying an array by a scalar (elementwise operation) \fi
	$$(\alpha \vec{x})_i = \alpha \, (\vec{x})_i$$
\item \ifita Sommare due array \`e anche operazione \emph{elementwise} \else
Summing two arrays (also elementwise)
\fi
$$(\vec{x}+\vec{y})_i = (\vec{x})_i + (\vec{y})_i$$
\item \ifita Moltiplicare due array (una riga per una colonna) non elementwise\else 
Product of two arrays (row by col) not elementwise\fi
$$\vec{x} \cdot \vec{y} = \sum_i (\vec{x})_i (\vec{y})_i$$
\end{itemize}}

\frame{\frametitle{\ifita Calcolo vettoriale (esercizi)\else Vector Calculus (exe)\fi}
\begin{itemize}
\item $\alpha = 3$,   $\vec{x}=[1 \quad 4 \quad 0]$, $\vec{y} = \left[ \begin{array}{c} 2 \\ 1 \\ -1 \end{array} \right]$, $\vec{z} = \left[ \begin{array}{c} 1 \\ 0 \\ 2 \end{array} \right]$
\item $\alpha \vec{y} = \left[ \begin{array}{c} 6 \\ 3 \\ -3 \end{array} \right]$
\quad 
$\vec{y} + \vec{z} = \left[ \begin{array}{c} 3 \\ 1 \\ 1 \end{array} \right]$
\vskip 2mm
\item $\vec{x} \cdot \vec{y} = 6 \quad \vec{x} \cdot \vec{z} = 1$
\item $\vec{y} \cdot \vec{z}=?$\quad\quad
\ifita a rigore non si pu\`o \else strictly speaking not possible \fi
\end{itemize}}

\frame{\frametitle{\ifita Matrici \else Matrices \fi}
\begin{itemize}
\item \ifita Matrice: ``array di array'' \else Matrix = array of arrays \fi
\item \ifita Dimensioni: numero di righe e numero di colonne \else Dimensions: number of rows and cools \fi\\
\ifita
$\hat{A}$ matrice con $n$ righe e $m$ colonne:
\else
$\hat{A}$ matrix with $n$ rows and $m$ cols:
\fi
 $dim(\hat{A})=[n,m]$
\item $dim(\hat{A})=[n,n]$?\quad
 $\hat{A}$ \ifita quadrata \else square matrix \fi
 \\
\end{itemize}
\vskip 4mm
$$\hat{A}= \left[ \begin{array}{ccc} 1 & 4 & 2\\ 4 & 1 &3 \\ 5&2&0 \end{array} \right]\quad
\hat{B}= \left[ \begin{array}{cc}  3 & 2\\ 1 & 2 \\ 5&0 \end{array} \right]\quad
\hat{C}= \left[ \begin{array}{ccc} 3 & 2 &0  \\ 1&4&1 \end{array} \right]$$
	$$dim(\hat{A})=[3,3] \quad dim(\hat{B})=[3,2] \quad dim(\hat{C})=[2,3]$$
	$$(\hat{A})_{1,2}=4 \quad (\hat{B})_{3,1}=5 \quad (\hat{C})_{2,3}=1 $$}
\frame{\frametitle{\ifita Tipi di matrici\else Kind of Matrices\fi}
\setstretch{1.4}
\begin{itemize}
\ifita
\item Diagonale di una matrice (quadrata): array degli elementi con uguale indice di riga e colonna
\item Matrice triangolare superiore (es. $\hat{A}$): matrice che ha elementi non nulli solo sulla diagonale e sopra di essa
\item Matrice triangolare inferiore (es. $\hat{C}$): matrice che ha elementi non nulli solo sulla diagonale e sotto di essa
\item Matrice diagonale (es. $\hat{B}$): matrice simultaneamente diagonale superiore ed inferiore
\else
\item Diagonal (of a square matrix): array of elements\\with same row and col index
\item Upper triangular matrix (ex. $\hat{A}$): elements under the diagonal are all zero
\item Lower triangular matrix (ex. $\hat{C}$): elements over the diagonal are all zero
\item Diagonal matrix (ex. $\hat{B}$): matrix at the same time\\upper and lower triangular
\fi
\end{itemize}
\vskip 4mm
$$\hat{A}= \left[ \begin{array}{ccc} 1 & 4 & 2\\ 0 & 1 &3 \\ 0&0&3 \end{array} \right]\quad
\hat{B}= \left[ \begin{array}{ccc} 1 & 0 & 0\\ 0 & 1 &0 \\ 0&0&3 \end{array} \right]\quad
\hat{C}= \left[ \begin{array}{ccc} 1 & 0 & 0\\ 5 & 1 &0 \\ 2&1&3 \end{array} \right]$$}	

\frame{\frametitle{\ifita Calcolo matriciale\else Matrix Calculus \fi}
\setstretch{1.2}
\begin{itemize}
\item \ifita Moltiplicare uno scalare per una matrice \`e operazione \emph{elementwise} 
\else
Matrix (elementwise) multiplication by a scalar
\fi
$$(\alpha \hat{A})_{ij} = \alpha \, (\hat{A})_{ij}$$
\item 
\ifita
Sommare due matrici \`e anche operazione \emph{elementwise}
\else
(Elementwise) Sum of two matrices
\fi
$$(\hat{A}+\hat{B})_{ij} = (\hat{A})_{ij} + (\hat{B})_{ij}$$
\item \ifita Moltiplicare due matrici non lo \`e!
\else
Matrix multiplication (not elementwise!)
\fi
$$(\hat{A} \cdot \hat{B})_{ij} = \sum_k (\hat{A})_{ik} (\hat{B})_{kj}$$
\item \ifita {\footnotesize $dim(\hat{A})=[n_A,m_A]$ ,  $dim(\hat{B})=[n_B,m_B]$,  $dim(\hat{A}\cdot \hat{B}) = [n_A,m_B]$\\
le matrici possono essere moltiplicate solo se $m_A=n_B$}
\else
{\footnotesize $dim(\hat{A})=[n_A,m_A]$ ,  $dim(\hat{B})=[n_B,m_B]$ , $dim(\hat{A}\cdot \hat{B}) = [n_A,m_B]$\\
matrices can be multiplied iff $m_A=n_B$}
\fi
\end{itemize}}

\frame{\frametitle{\ifita Calcolo matriciale (esercizi) \else Matrix Calculus (exe) \fi}
\setstretch{1.4}
$$\alpha = 3 \quad \hat{A}= \left[ \begin{array}{ccc} 1 & -1 & 2\\ 1 & 0 & 2 \\ 1 & 2 & 0 \end{array} \right]\quad
\hat{B}= \left[ \begin{array}{ccc} 1 & 1 & 3\\ 2 & 1 & 0 \\ 1 & -1 & 2 \end{array} \right]$$
$$\alpha \cdot \hat{A}= \left[ \begin{array}{ccc} 3 & -3 & 6\\ 3 & 0 & 6 \\ 3 & 6 & 0 \end{array} \right]
\quad	
\hat{A} \cdot \hat{B}= \left[ \begin{array}{ccc}  1 & -2  & 7\\ 3 &-1  & 7 \\ 5 & 3 &3  \end{array} \right]$$
\begin{itemize}
\item \ifita Esercizio: verificare che \else Exercise: Check \fi $\hat{A}\cdot\hat{B}\neq \hat{B}\cdot\hat{A}$
\end{itemize}}

\frame{\frametitle{\ifita La matrice identit\`a \else Identity Matrix \fi}
\setstretch{1.2}
\begin{itemize}
\item \ifita Matrice \emph{identit\`a} $\hat{I}$: matrice diagonale, tutti uno sulla diagonale
\else
Identity matrix $\hat{I}$: diagonal matrix, ones on the diagonal
\fi
\item \ifita L'identit\`a \`e l'elemento neutro del prodotto:
\else
Identity as neutral element
\fi
 $\hat{I}\cdot \hat{A} = \hat{A} \cdot \hat{I} = \hat{A}$
$$\left[ \begin{array}{ccc}  1 & 0  & 0\\ 0 &1  & 0 \\ 0 & 0 &1  \end{array} \right]
\cdot \left[ \begin{array}{ccc}  3 & 1  & 2\\ 1 &2  & 1 \\ 2 & 0 &1  \end{array} \right]
= \left[ \begin{array}{ccc}  3 & 1  & 2\\ 1 &2  & 1 \\ 2 & 0 &1  \end{array} \right]$$
$$\left[ \begin{array}{ccc}  3 & 1  & 2\\ 1 &2  & 1 \\ 2 & 0 &1  \end{array} \right]
\cdot
\left[ \begin{array}{ccc}  1 & 0  & 0\\ 0 &1  & 0 \\ 0 & 0 &1  \end{array} \right]
= \left[ \begin{array}{ccc}  3 & 1  & 2\\ 1 &2  & 1 \\ 2 & 0 &1  \end{array} \right]$$
\end{itemize}}

\frame{\frametitle{\ifita La matrice inversa \else Inverse Matrix \fi}
\setstretch{1.2}
\begin{itemize}
\item 
\ifita Dato uno scalare $k$, il suo \emph{reciproco} $k^{-1}$ \`e un numero tale che
\else
Given scalar $k$, its \emph{reciprocal} $k^{-1}$ is a number such that
\fi
 $$k \cdot k^{-1} = k^{-1} \cdot k = 1$$
\item \ifita
Data una matrice $\hat{A}$, la sua \emph{inversa} $\hat{A}^{-1}$ \`e una matrice tale che 
\else
Given matrix $\hat{A}$, its \emph{inverse} $\hat{A}^{-1}$ is such that \fi
$$\hat{A} \cdot \hat{A}^{-1} = \hat{A}^{-1} \cdot \hat{A} = \hat{I}$$
\item \ifita Invertire una matrice non \`e banale (e gli algoritmi che studieremo ci serviranno proprio a questo)
\else
Matrix inversion is not trivial, the algorithms we will study are basically intended to do that
\fi
$$\hat{A}=\left[ \begin{array}{ccc}  1 & 1  & 1\\ 0 &1  & 1 \\ 1 & 0 &1  \end{array} \right]
\quad
\hat{A}^{-1}=
\left[ \begin{array}{ccc}  1 & -1  & 0\\ 1 &0  & -1 \\ -1 & 1 &1  \end{array} \right]$$
\end{itemize}}

\frame{\frametitle{\ifita Rappresentazione matriciale di un sistema lineare \else
Matrix Representation of a Linear System \fi    
}
\setstretch{1.6}
$$\hat{A}=\left[ \begin{array}{ccc}  1 & -1 & 2\\ 2  & -1 & -1 \\ 1 & 1 & 1  \end{array} \right]
\quad
\vec{x} = 
\left[ \begin{array}{c}  x_1 \\  x_2 \\  x_3 \end{array} \right]
\quad
\vec{b} = 
\left[ \begin{array}{c}  2 \\  0 \\  3 \end{array} \right]$$
$$\hat{A}\cdot\vec{x}  = \vec{b}$$
$$\left\{ \begin{array}{c}  x_1 - x_2 + 2 x_3 = 2 \\ 2x_1-x_2-x_3=0  \\ x_1+x_2+x_3 = 3  \end{array} \right.
\quad
\vec{x} = \hat{A}^{-1}\cdot \vec{b} = \left[ \begin{array}{c}  1 \\  1 \\  1 \end{array} \right]$$
}
\end{document}
