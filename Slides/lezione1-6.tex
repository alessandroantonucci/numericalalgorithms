\documentclass[professionalfonts,handout]{beamer}
\usepackage{circuitikz}
\usepackage[familydefault,light]{Chivo} 
\usepackage[T1]{fontenc}
\usenavigationsymbolstemplate{}
\usepackage[]{hyperref}
\usepackage{tikz,pgf,pgfarrows,pgfnodes,pgfbaseimage}
\graphicspath{{./Pics/}}
\usetikzlibrary{shapes}
\usepackage{setspace}
\newcommand{\evi}[1]{{\colorbox{yellow!50}{{#1}}}}
\newcommand{\exe}[1]{{\color{black!50}{{#1}}}}
\newcommand{\kw}[1]{{\colorbox{black!30}{\color{white}{#1}}}}
\tikzstyle{nd}=[circle,draw=black,thick,minimum size=.8cm,inner sep=1pt]
\setbeamercovered{transparent}
\usetheme{Singapore}
\tikzstyle{nodo}=[ellipse,draw=black!60,fill=black!10,line width=.7pt,minimum width=.7cm,minimum height=.4cm]
\usecolortheme[named=gray]{structure}
\setbeamercolor{block title}{bg=black!20,fg=black}
\setbeamercolor{block body}{bg=black!10,fg=black}

\title{Algoritmi Numerici (Parte I)}
\subtitle{[Lezione 6] Operazioni aritmetiche}
\author{Alessandro Antonucci\\{\tt alessandro.antonucci@supsi.ch}}
\date{\tiny\url{https://colab.research.google.com/drive/10HxMjR7A0Qa8X3cRe2ZCgO-rtRyYvVKY}}
%%%%%%%%%%%%%%%%%%%%%%%%%%%%
\begin{document}
\maketitle
\frame{\frametitle{Somma}
\begin{itemize}
\item Sistema a 4 bit con complemento a 2
\item Somma con circuiti con porte Booleane (\href{https://upload.wikimedia.org/wikipedia/commons/thumb/9/92/Halfadder.gif/220px-Halfadder.gif}{link}) \\ \exe{(es. $0+0=1+1=0$ e $0+1=1+0=1$ \`e XOR)} 
\item $0010 + 0011 = 0101$ \exe{$2+3=5$} 
\item $0010 + 1100 = 1110$ \exe{$2+(-4)=-2$}
\item $0011 + 0110 = 1001$ \exe{$3+6 \neq -7$} overflow!
\item $1100 + 1010 = 0110$ \exe{$-4-6 \neq 10$} underflow!
\end{itemize}
\begin{center}
\emph{ \evi{Over/under flow se il primo bit \`e uguale}
\evi{nei due addendi, ma diverso nella somma}}
\end{center}
\begin{itemize}
\item $1100+1100=1000$ \exe{$-4-4=-8$} \\(giusto anche con riporto)
\end{itemize}}

\frame{\frametitle{Sottrazione (e precisione multipla)}
\begin{itemize}
\item Somma di due numeri a 4 bit:
\begin{itemize}
\item Due addendi (4+4 bit) 
\item Carry (C, 1 bit) si attiva per il riporto \exe{$1+1=1$ \`e AND}
\item Overflow (V, 1 bit) si attiva per identificare over/under
\end{itemize}
\item Numeri piu grandi
\begin{itemize}
\item Precisione multipla (tante locazioni affiancate)
\item Si inizializza $C=0$ (non importa se alla fine $C=1$)
\item Risultato corretto solo se $V=0$
\end{itemize}
\item Sottrazione
\begin{itemize}
\item a - b = a + (-b)
\item Per scrivere $-b$, scrivo $b$ in base 2 e poi cambio segno \\ (ricopio da dx finch\'e non trovo $1$, scrivo quello, poi nego)
\end{itemize}
\end{itemize}}

\frame{\frametitle{Moltiplicazione e Divisione}
\begin{itemize}
\item Moltiplicazioni in colonna = somme numeri spostati a sx
\item ASL (arithmetic shift left) sposto i bit a sx (uno zero a dx)
\item Es. ASR(0011)=0110 
\item \`E moltiplicazione per 2 (se $V=0$ vale anche con complemento a 2)
\item Divisione: analoga ma sposto a dx
\item ASR (arithmetic shift right), divisione (intera) per 2
\end{itemize}}

\frame{\frametitle{Aritmetica float}
\begin{itemize}
\item Dati due numeri (float)
\begin{itemize}
\item $a = m_1 \cdot b^q$ 
\item	$b = m_2 \cdot b^p$ ($q>p$)
\end{itemize}
\item Posso calcolare
\begin{itemize}
\item $a+b = (m_1 + m_2 \cdot b^{p-q}) \cdot b^q$
\item $a \cdot b = (m_1 \cdot m_2) \cdot b^{q+p}$
\item $a / b= (m_1 / m_2) \cdot b^{q-p}$
\end{itemize}
\item Le operazioni sulla mantissa introducono un errore\\
(riducono le cifre significative)
\end{itemize}}

\frame{\frametitle{Under/over flow con numeri float} 
Formati float-like caratterizzati da:
\begin{itemize}
\item $m_1$ numero pi\`u piccolo rappresentabile
\item $M_1$ numero pi\`u grande.
\end{itemize}
Quando inserisco un numero $x$, il sistema restituisce
\begin{itemize}
\item overflow se $x$ esterno a $[-M_1,M_1]$
\item underflow se interno a $(-m_1,m_1)$ (approssimato con lo zero)
\end{itemize}
Nel formato float, se tutti i bit dell'esponente sono uno, casi speciali
\vskip 2mm
Es. $0|11111111|00000000000000000000000\to + \infty$
Es. $0|11111111| 00000000001000001000000 \to NaN$}
\end{document}
