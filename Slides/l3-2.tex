\documentclass[professionalfonts]{beamer}
\newif\ifita
%\itatrue % comment out to hide answers
\itafalse
\usepackage[familydefault,light]{Chivo} 
\usepackage[T1]{fontenc}
\usenavigationsymbolstemplate{}
\usepackage[]{hyperref}
\usepackage{tikz,pgf,pgfarrows,pgfnodes,pgfbaseimage}
\graphicspath{{./Pics/}}
\usetikzlibrary{shapes}
\usepackage{setspace}
\newcommand{\evi}[1]{{\colorbox{yellow!50}{{#1}}}}
\newcommand{\exe}[1]{{\color{black!50}{{#1}}}}
\newcommand{\kw}[1]{{\colorbox{black!30}{\color{white}{#1}}}}
\tikzstyle{nd}=[circle,draw=black,thick,minimum size=.8cm,inner sep=1pt]
\setbeamercovered{transparent}
\usetheme{Singapore}
\tikzstyle{nodo}=[ellipse,draw=black!60,fill=black!10,line width=.7pt,minimum width=.7cm,minimum height=.4cm]
\usecolortheme[named=gray]{structure}
\setbeamercolor{block title}{bg=black!20,fg=black}
\setbeamercolor{block body}{bg=black!10,fg=black}

%%%%%%%%%%%%%%%%%%%%%
\ifita
\title{Algoritmi Numerici (Parte III)}
\subtitle{[Lezione 1] Sistemi Triangolari}
\else
\title{Numerics (Part III)}
\subtitle{[Lecture 1] Triangular Linear Systems}
\fi
\date{}
\author{Alessandro Antonucci\\{\tt alessandro.antonucci@supsi.ch}}
\begin{document}
\maketitle
\setstretch{1.4}
%%%%%%%%%%%%%%%%%%%%%
\frame{\frametitle{\ifita Rappresentazione matriciale di un sistema lineare \else Matrix Representation of a Linear System \fi}
\setstretch{1.6}
$$\hat{A}=\left[ \begin{array}{ccc}  1 & -1 & 2\\ 2  & -1 & -1 \\ 1 & 1 & 1  \end{array} \right]
\quad
\vec{x} = 
\left[ \begin{array}{c}  x_1 \\  x_2 \\  x_3 \end{array} \right]
\quad
\vec{b} = 
\left[ \begin{array}{c}  2 \\  0 \\  3 \end{array} \right]$$
\centering
\begin{columns}
\begin{column}[T]{0.3\textwidth} 
$$\left\{ \begin{array}{c}  x_1 - x_2 + 2 x_3 = 2 \\ 2x_1-x_2-x_3=0  \\ x_1+x_2+x_3 = 3  \end{array} \right.$$
\end{column}
\begin{column}[T]{0.3\textwidth} 
\vskip 10mm
$$\hat{A}\cdot\vec{x}  = \vec{b}$$
\end{column}
\end{columns}}

\frame{\frametitle{\ifita Rappr. matriciale di un sistema tringolare \else Matrix Representation of a Triangular System \fi
}
\setstretch{1.6}
$$\hat{A}=\left[ \begin{array}{ccc}  1 & -1 & 2\\ 0  & -1 & -1 \\ 0 & 0 & 1  \end{array} \right]
\quad
\vec{x} = 
\left[ \begin{array}{c}  x_1 \\  x_2 \\  x_3 \end{array} \right]
\quad
\vec{b} = 
\left[ \begin{array}{c}  2 \\  0 \\  3 \end{array} \right]$$
\centering
\begin{columns}
\begin{column}[T]{0.3\textwidth} 
$$\left\{ \begin{array}{c}  x_1 - x_2 + 2 x_3 = 2 \\ \phantom{2x_1}-x_2-x_3=0  \\ \phantom{x_1+x_2+}x_3 = 3  \end{array} \right.$$
\end{column}
\begin{column}[T]{0.7\textwidth} 
$$\left\{ \begin{array}{ll}  x_1 =& 2 + x_2 - 2 x_3 = 2-3-6=-7  \\ x_2 = &-x_3 = -3  \\ x_3 =& 3  \end{array} \right.$$
\end{column}
\end{columns}}

\frame{\frametitle{\ifita Algoritmo di sostituzione \else Solving by Substitution \fi}
\begin{itemize}
\item INPUT: \ifita sistema triangolare \else triangular system \fi
$\hat{A}$, $\vec{b}$ \ifita (con $\hat{A}$ triangolare) \else ($\hat{A}$ triangular)\fi
\item OUTPUT: \ifita soluzione \else solution \fi $\vec{x}^*$ \ifita tale che \else such that \fi $\hat{A}\cdot\vec{x}^* = \vec{b}$
\end{itemize}
\vskip 3mm
\ifita
Sostituzione dal basso verso l'alto, da dx verso sx (fermandosi sulla diagonale)
\else
Bottom-up substitution + right-left\\ (stop just before the diagonal)
\fi
\vskip 3mm
\begin{columns}
\begin{column}[T]{0.4\textwidth}
{\tt n} \ifita dimensione matrice \else matrix dimension \fi
\vskip 1mm
{\tt for i=n .. 1}
\vskip 1mm
{\tt \quad \quad for j=n .. i}
\vskip 1mm
{\tt \quad \quad \quad \quad x(i)= ...}
\vskip 1mm
{\tt \quad \quad end}
\vskip 1mm
{\tt end}
\end{column}
\begin{column}[T]{0.5\textwidth} 
\centering\it
\ifita 
vettore delle soluzioni $\vec{x}^*$\\ricavato partendo\\dall'ultimo\\elemento
\else
solution vector $\vec{x}^*$\\obtained by starting from the last element
\fi
\end{column}
\end{columns}}


\frame{\frametitle{Manual $5 \times 5$ in Python $\ldots$}
\lstinputlisting[
language=Python,
firstline=20,
lastline=30
]{e0.py}}

\frame{\frametitle{\ifita Sostituzione \else Substitution \fi in Python $\ldots$}
    \lstinputlisting[
    language=Python,
    firstline=0,
    lastline=10
    ]{subst.py}}


\frame{\frametitle{\ifita Complessit\`a algoritmo di sostituzione \else Computational Complexity of Substitution\fi}
\begin{itemize}
\ifita
\item Una divisione per ogni equazione ($n$ divisioni)
\item Una sottrazione ed una moltiplicazione per ogni elemento sopra-diagonale
\item Numero elementi sopra-diagonali $=\frac{n^2-n}{2}$
\item Numero operazioni $=n+2\cdot \frac{n^2-n}{2}=n^2$
\item Complessit\`a quadratica $O(n^2)$
\else
\item One division for each equation ($n$ divisions)
\item One subtraction and one multiplication for each upper-diagonal element
\item Upper-diagonal Elements $=\frac{n^2-n}{2}$
\item Number of Basic Operations $=n+2\cdot \frac{n^2-n}{2}=n^2$
\item Quadratic Complexity $O(n^2)$
\fi
\end{itemize}}
\end{document}
