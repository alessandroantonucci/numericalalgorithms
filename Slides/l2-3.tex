\documentclass[professionalfonts]{beamer}
\newif\ifita
%\itatrue
\itafalse
\usepackage[familydefault,light]{Chivo} 
\usepackage[T1]{fontenc}
\usenavigationsymbolstemplate{}
\usepackage[]{hyperref}
\usepackage{tikz,pgf,pgfarrows,pgfnodes,pgfbaseimage}
\graphicspath{{./Pics/}}
\usetikzlibrary{shapes}
\usepackage{setspace}
\newcommand{\evi}[1]{{\colorbox{yellow!50}{{#1}}}}
\newcommand{\exe}[1]{{\color{black!50}{{#1}}}}
\newcommand{\kw}[1]{{\colorbox{black!30}{\color{white}{#1}}}}
\tikzstyle{nd}=[circle,draw=black,thick,minimum size=.8cm,inner sep=1pt]
\setbeamercovered{transparent}
\usetheme{Singapore}
\tikzstyle{nodo}=[ellipse,draw=black!60,fill=black!10,line width=.7pt,minimum width=.7cm,minimum height=.4cm]
\usecolortheme[named=gray]{structure}
\setbeamercolor{block title}{bg=black!20,fg=black}
\setbeamercolor{block body}{bg=black!10,fg=black}

\ifita
\title{Algoritmi Numerici (Parte II)}
\subtitle{[Lezione 5] Convergenza}
\else
\title{Numerics (Part II)}
\subtitle{[Lecture 5] Convergence}
\fi
\date{}
\author{Alessandro Antonucci\\{\tt alessandro.antonucci@supsi.ch}}
\begin{document}
\maketitle
\setstretch{1.4}

\setstretch{1.4}
\frame{\frametitle{\ifita Convergenza \else Convergence \fi} 
\ifita
Una successione di valori $x_0,x_1,x_2,\ldots$ converge verso un valore $x^*$ se la distanza/errore
$\epsilon_k := \mid x_k-x^*\mid$ tende sempre pi\`u ad avvicinarsi a zero con il
crescere di $k$
$$\lim_{k\to+\infty} \epsilon_k = 0$$
Un algoritmo iterativo per la ricerca degli zeri di
una funzione genera una successione di valori $x_0,x_1,x_2,\ldots$,\\
tale che, quando converge, lo fa verso uno zero $x^*$ di $f$
\else
A sequence
$x_0,x_1,x_2,\ldots$ \evi{converges} to $x^*$ if the distance/error
$\epsilon_k := \mid x_k-x^*\mid$ tends to zero for increasing $k$
$$\lim_{k\to+\infty} \epsilon_k = 0$$
When converging, an iterative algorithm to find the zero of $f$ outputs a sequence $x_0,x_1,x_2,\ldots$,
converging to $x^*$
\fi
}

\frame{\frametitle{\ifita Ordine di convergenza \else Convergence Order \fi}
\ifita
Un algoritmo
iterativo ha ordine di convergenza $p$ se esistono due numeri $C\geq
0$ e $p\geq 0$ tali che $$\lim_{k\rightarrow\infty}
\frac{e_{k+1}}{e_k^p}=C,$$ ovvero $|x_{k+1}-x^*|<C |x_k-x^*|^p$
\vskip 2mm
Se $p=1$ si dice che l'ordine di convergenza
\`e lineare\\ superlineare con $1<p<2$, quadratico con $p=2$
\else
An iterative algorithm has \evi{convergence order} $p$ if there are two numbers $C\geq
0$ e $p\geq 0$ such that $$\lim_{k\rightarrow\infty}
\frac{e_{k+1}}{e_k^p}=C,$$ that means $|x_{k+1}-x^*|<C |x_k-x^*|^p$
\vskip 2mm
If $p=1$ we say that the convergence is linear, \\ superlinear with $1<p<2$, quadratic with $p=2$
\fi
}

\frame{\frametitle{\ifita Convergenza dei vari algoritmi \else Algorithms Convergence \fi}
\begin{itemize}
\ifita
\item L'algoritmo della bisezione e le sue varianti convergono linearmente ($p=1$)
\item L'algoritmo della secante converge superlinearmente\\
($p=\frac{1+\sqrt{5}}{2} \simeq 1.618$)
\item L'algoritmo della tangente converge quadraticamente ($p=2$)
\else
\item Bisection and its variants converge linearly ($p=1$)
\item Secant method converges superlinearly\\
($p=\frac{1+\sqrt{5}}{2} \simeq 1.618$)
\item Tangent method quadratically ($p=2$)
\fi
\end{itemize}}

\frame{\frametitle{\ifita Convergenza su punti a tangenza orizzontale\else Slow Convergence on zero-derivative points\fi}
\begin{itemize}
\ifita
\item Se $f'(x^*)=0$ (zero con tangente orizzontale), l'algoritmo della tangente ``rallenta'' e la convergenza \`e lineare e non quadratica
\item Es. con $f(x)=x^2$, $f(x)=2x$ allora $x^*=0$ e $f'(x^*)=0$.
\else
\item If $f'(x^*)=0$ (zero of $f$ with horizontal tangent), tangent method becomes "slow" and convergence is linear (instead of quadratic)
\item Ex. $f(x)=x^2$, $f'(x)=2x$ then $x^*=0$ and $f'(x^*)=0$.
\fi
\end{itemize}
\vskip 1mm
\centering
	\small
\begin{tabular}{cccc}
$k$&$x_k$&$f(x_k)$&$f'(x_k)$\\
	\hline
$0$	& $1$	&$1$&$2$\\
$1$	& $1-\frac{1}{2}=\frac{1}{2}$&$\frac{1}{4}$&$1$\\
$2$	&$\frac{1}{2}-\frac{\frac{1}{4}}{1}=\frac{1}{4}$&$\frac{1}{16}$&$\frac{1}{2}$\\
$3$	&$\frac{1}{4}-\frac{\frac{1}{16}}{\frac{1}{2}}=\frac{1}{8}$\\
\end{tabular}
\ifita
\vskip 1mm	\emph{Ogni iterazione dimezza l'errore, convergenza lineare!}
\vskip 1mm la stessa cosa succede con l'algoritmo della secante
\else
\vskip 1mm	\emph{Each iteration makes the error (roughly)\\half of the previous error, linear convergence!}
\begin{itemize}
\item The same happens with secant method
\end{itemize}
\fi
}

\frame{\frametitle{\ifita In Pratica \else In practice \fi $\ldots$}
\begin{itemize}
\ifita
\item Itera l'algoritmo e genera $x_0,x_1,\ldots,x_n$
\item Se la sequenza converge (a $x^*$) calcola\\la sequenza degli errori $\epsilon_0,\epsilon_1,\ldots,\epsilon_n$\\con $\epsilon_j:=|x_j-x^*|$
\item Calcola la sequenza dei rapporti fra gli errori
$\gamma_0,\gamma_1,\ldots,\gamma_n$ con $\gamma_j:=\epsilon_j/\epsilon_{j-1}$
\item Se la sequenza dei rapporti va a zero, allora la convergenza \`e pi\`u che lineare, altrimenti \`e lineare
\item Per capire se la convergenza \`e quadratica o solo superlineare occorre valutare anche
$\gamma_j^p:=\epsilon_j/\epsilon_{j-1}^p$
\else
\item Iterate the algorithm to generate $x_0,x_1,\ldots,x_n$
\item If the sequence converges (to $x^*$) compute\\the sequence of errors $\epsilon_0,\epsilon_1,\ldots,\epsilon_n$\\where $\epsilon_j:=|x_j-x^*|$
\item Compute the sequence of error ratios
$\gamma_0,\gamma_1,\ldots,\gamma_n$ where $\gamma_j:=\epsilon_j/\epsilon_{j-1}$
\item If the ratio sequence goes to zero,\\then convergence is superlinear, linear otherwise
\item To decide whether convergence is quadratic or just superlinear, you also need 
$\gamma_j^p:=\epsilon_j/\epsilon_{j-1}^p$ (for various $p$)
\fi
\end{itemize}


}



\end{document}
