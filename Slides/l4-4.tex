\documentclass[professionalfonts]{beamer}
\usepackage[english]{babel}
\usepackage{verbatim}
\usepackage{graphicx}
\usepackage{pgfplots}
\usetikzlibrary{automata,topaths}
\pgfplotsset{width=7cm,compat=1.8}
\usepackage{pgfplotstable}
\pgfmathsetseed{1138} % set the random seed
\pgfplotstableset{ % Define the equations for x and y
    create on use/x/.style={create col/expr={42+2*\pgfplotstablerow}},
    create on use/y/.style={create col/expr={(0.6*\thisrow{x}+130)+5*rand}}
}
% create a new table with 30 rows and columns x and y:
\pgfplotstablenew[columns={x,y}]{30}\loadedtable

\usepackage[familydefault,light]{Chivo} 
\usepackage[T1]{fontenc}
\usenavigationsymbolstemplate{}
\usepackage[]{hyperref}
\usepackage{tikz,pgf,pgfarrows,pgfnodes,pgfbaseimage}
\graphicspath{{./Pics/}}
\usetikzlibrary{shapes}
\usepackage{setspace}
\newcommand{\evi}[1]{{\colorbox{yellow!50}{{#1}}}}
\newcommand{\exe}[1]{{\color{black!50}{{#1}}}}
\newcommand{\kw}[1]{{\colorbox{black!30}{\color{white}{#1}}}}
\tikzstyle{nd}=[circle,draw=black,thick,minimum size=.8cm,inner sep=1pt]
\setbeamercovered{transparent}
\usetheme{Singapore}
\tikzstyle{nodo}=[ellipse,draw=black!60,fill=black!10,line width=.7pt,minimum width=.7cm,minimum height=.4cm]
\usecolortheme[named=gray]{structure}
\setbeamercolor{block title}{bg=black!20,fg=black}
\setbeamercolor{block body}{bg=black!10,fg=black}

%%%%%%%%%%%%%%%%%%%%%
\title{Algoritmi Numerici (Parte IV)}
\subtitle{Lezione 4: regressione lineare}
\author{Alessandro Antonucci\\{\tt alessandro.antonucci@supsi.ch}}
\date{}
%%%%%%%%%%%%%%%%%%%%%%%%%%%%
\begin{document}
\maketitle
\setbeamercovered{}

\frame{\frametitle{Motivazioni}
\begin{columns}
\begin{column}{0.4\textwidth}
\begin{itemize}
\item Dati su peso e altezza di 30 persone
\item Interpolazione poco significativa
\item Regressione (lineare) per estrapolare conoscenza dai dati
\end{itemize}
\end{column}
\begin{column}{0.6\textwidth}
\begin{tikzpicture}
\begin{axis} [
xlabel     = Peso (kg),ylabel     = Altezza (cm),
axis lines = left,clip= false,xmin = 40,xmax = 105,ymin = 150,ymax = 200.]
\addplot [only marks] table {\loadedtable};
\end{axis}
\end{tikzpicture}   
\end{column}
\end{columns}}

\frame{\frametitle{Motivazioni}
\begin{columns}
\begin{column}{0.4\textwidth}
\begin{itemize}
\item Dati su peso e altezza di 30 persone
\item Interpolazione poco significativa
\item Regressione (lineare) per estrapolare conoscenza dai dati
\end{itemize}
\end{column}
\begin{column}{0.6\textwidth}
\begin{tikzpicture}
\begin{axis} [
xlabel     = Peso (kg),ylabel     = Altezza (cm),
axis lines = left,clip= false,xmin = 40,xmax = 105,ymin = 150,ymax = 200.]
\addplot [mark=*] table {\loadedtable};
\end{axis}
\end{tikzpicture}   
\end{column}
\end{columns}}

\frame{\frametitle{Motivazioni}
\begin{columns}
\begin{column}{0.4\textwidth}
\begin{itemize}
\item Dati su peso e altezza di 30 persone
\item Interpolazione poco significativa
\item Regressione (lineare) per estrapolare conoscenza dai dati
\end{itemize}
\end{column}
\begin{column}{0.6\textwidth}
\begin{tikzpicture}
\begin{axis} [
xlabel     = Peso (kg),ylabel     = Altezza (cm),
axis lines = left,clip= false,xmin = 40,xmax = 105,ymin = 150,ymax = 200.]
\addplot [only marks,mark=*] table {\loadedtable};
%{\onslide<2->\addplot [mark=*] table {\loadedtable};}
\addplot [only marks] table {\loadedtable};
\addplot [no markers, thick, red]
  table [y={create col/linear regression={y=y}}] {\loadedtable}
  node [anchor=west] {};
\end{axis}
\end{tikzpicture}   
\end{column}
\end{columns}}

\frame{\frametitle{Regressione lineare}
\begin{columns}
\begin{column}{0.5\textwidth}
\begin{itemize}
\item Dati $n$ punti $\{ (x_i,y_i) \}_{i=1}^n$, cerco una funzione che descriva la relazione fra $x$ e $y$
\item Relazione lineare: $y = m x + q$, ma come trovare $m$ e $q$? 
\item
Distanza sulla verticale fra punto e retta $\epsilon_i := y_i - mx_i - q$
\end{itemize}
\end{column}
\begin{column}{0.5\textwidth}
\begin{tikzpicture}[domain=0:4]
\draw[-] (0.5,1.6)--(3,3.3);
\draw[->] (-0.2,0.9) -- (4.2,0.9) node[right] {$x$};
\draw[->] (0,0.5) -- (0,4.2) node[above] {$y$};
\draw[color=black,fill=black,very thick] (2,4) circle (0.07cm) node[above,yshift=5pt,fill=white]{$(x_i,y_i)$};
\draw[color=black,fill=black,very thick] (2,2.65) circle (0.07cm);
\draw[dotted](2,4)--(2,2.65);
\draw (2.8,2.15) node[]{$(x_i,m x_i+q)$};
\draw (1.8,3.3) node[]{$\epsilon_i$};
\end{tikzpicture}
\end{column}
\end{columns}
}


\frame{\frametitle{Regressione lineare}
\begin{columns}
\begin{column}{0.5\textwidth}
\begin{itemize}
\item Minimizzare $\sum_i \epsilon_i$ ?
\item No, $\epsilon_i<0$ se punto sotto retta
\item Meglio minimizzare $\sum_i \epsilon_i^2$
\item Ottimizzazione funzione di due variabili ($m$ e $q$)
\end{itemize}
\end{column}
\begin{column}{0.5\textwidth}
\begin{tikzpicture}[domain=0:4]
\draw[-] (0.5,1.6)--(3,3.3);
\draw[->] (-0.2,0.9) -- (4.2,0.9) node[right] {$x$};
\draw[->] (0,0.5) -- (0,4.2) node[above] {$y$};
\draw[color=black,fill=black,very thick] (2,4) circle (0.07cm); \draw[dotted](2,4)--(2,2.65);
\draw[color=black,fill=black,very thick] (1.5,3.1) circle (0.07cm); 
\draw[color=black,fill=black,very thick] (1.75,2.0) circle (0.07cm); 
\draw[color=black,fill=black,very thick] (2.5,3.3) circle (0.07cm); 
\draw[dotted](2.5,3.3)--(2.5,3);
\draw[dotted](1.5,3.1)--(1.5,2.3);
\draw[dotted](1.75,2.0)--(1.75,2.5);

\end{tikzpicture}
\end{column}
\end{columns}
\begin{center}
$F(m,q) = \sum_{i=1}^n \epsilon_i^2 = \sum_{i=1}^n (y_i-m x_i -q)^2$
\end{center}
}

\frame{\frametitle{Ottimizzazione}
\begin{itemize}
\item Minimizzare/massimizzare $f(x)$? Pongo $\frac{\mathrm{d}f}{\mathrm{d}x}=0$
\item Minimizzare/massimizzare $F(m,q)$? 
$\frac{\mathrm{d}F}{\mathrm{d}m}=\frac{\mathrm{d}F}{\mathrm{d}q}=0$
\end{itemize}
$\left\{\begin{array}{lll}
\frac{\mathrm{d}F}{\mathrm{d}q}&=-2 \sum_i (y_i-mx_i-q)&= 0\\
\frac{\mathrm{d}F}{\mathrm{d}m}&=-2 \sum_i x_i (y_i-mx_i-q)&= 0
\end{array}\right.$ 

$\left\{\begin{array}{l}
n \overline{y} - nm\overline{x} - nq= 0\\
\sum_i x_i y_i -m \sum_i x_i^2 -q\sum_i x_i= 0
\end{array}\right.$ 

$\left\{\begin{array}{l}
q = \overline{y}-m\overline{x} \\
\sum_i x_i y_i -m \sum_i x_i^2 =n \overline{x} (\overline{y}-m\overline{x})
\end{array}\right.$ 

$\left\{\begin{array}{l}
q = \overline{y}-m\overline{x} \\
\sum_i x_i y_i -n \overline{x} \cdot \overline{y} = m (\sum_i x_i^2 -n \overline{x}^2)
\end{array}\right.$ 

\vskip 2mm

con 
$\overline{x} = \frac{\sum_i x_i}{n}$ e
$\overline{y} = \frac{\sum_i y_i}{n}$
}

\frame{\frametitle{Regressione Lineare (formule)}
\Large
La retta \`e $y=mx+q$ 
\vskip 5mm
con
\vskip 5mm
$\left\{\begin{array}{l}
q = \overline{y}-m\overline{x} \\
m = \frac{\sum_i x_i y_i -n \overline{x} \cdot \overline{y}}{
\sum_i x_i^2 -n \overline{x}^2}\end{array}\right.$
\vskip 5mm
e
\vskip 5mm
$\overline{x} = \frac{\sum_i x_i}{n}$ e
$\overline{y} = \frac{\sum_i y_i}{n}$


}




\end{document}


\end{document}
\frame{\frametitle{Regressione lineare}
\begin{itemize}
\item $\sum_i \epsilon_i^2$ con $\epsilon_i = y_i - m x_i +q$ \`e una funzione di $m$ e $q$
\item $F(m,q):= \sum_{i=1}^n (y_i - m x_i - q)^2$
\item Cerco valore/i di $m$ e $q$ per cui la derivata rispetto a $m$ ed a $q$ sono nulle
\begin{itemize}
\item $\partial_q F = -2 \sum_i (y_i-m x_i -q) = 0$
\item $\partial_m F = -2 \sum_i x_i (y_i -m x_i -q) = 0$
\end{itemize}
\end{itemize}
}


\frame{


\frame{\frametitle{a}
\begin{equation}
\left\{ \begin{array}{l} \sum_i y_i - m \sum_i x_i - n q = 0\\ \sum_i x_i y_i - m \sum_i x_i^2 - n q \end{array} \right.
\end{equation}

Definisco i valori medi $\overline{x} := \sum_i x_i / n$ e $\overline{y} := \sum_i y_i$.

\begin{equation}
\left\{ \begin{array}{l} \overline{y} = m \overline{x} + q \\ \sum_i x_i y_i - m \sum_i x_i^2 - n \overline{y} - n m \overline{x} \end{array} \right.
\end{equation}

da cui
\begin{equation}
\left\{ \begin{array}{l} 
q = \overline{y} - m \overline{x} + q \\
m = \frac{\sum_i x_i y_i - m \sum_i x_i^2 }{- n \overline{y} - n m \overline{x} }
\end{array} \right.
\end{equation}
}
\end{document}
