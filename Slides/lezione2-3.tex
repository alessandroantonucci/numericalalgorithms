\documentclass[professionalfonts]{beamer}
\usepackage[familydefault,light]{Chivo} 
\usepackage[T1]{fontenc}
\usenavigationsymbolstemplate{}
\usepackage[]{hyperref}
\usepackage{tikz,pgf,pgfarrows,pgfnodes,pgfbaseimage}
\graphicspath{{./Pics/}}
\usetikzlibrary{shapes}
\usepackage{setspace}
\newcommand{\evi}[1]{{\colorbox{yellow!50}{{#1}}}}
\newcommand{\exe}[1]{{\color{black!50}{{#1}}}}
\newcommand{\kw}[1]{{\colorbox{black!30}{\color{white}{#1}}}}
\tikzstyle{nd}=[circle,draw=black,thick,minimum size=.8cm,inner sep=1pt]
\setbeamercovered{transparent}
\usetheme{Singapore}
\tikzstyle{nodo}=[ellipse,draw=black!60,fill=black!10,line width=.7pt,minimum width=.7cm,minimum height=.4cm]
\usecolortheme[named=gray]{structure}
\setbeamercolor{block title}{bg=black!20,fg=black}
\setbeamercolor{block body}{bg=black!10,fg=black}

\title{Algoritmi Numerici (Parte II)}
\subtitle{[Lezione 3] Convergenza}
\author{Alessandro Antonucci\\{\tt alessandro.antonucci@supsi.ch}}
\date{\tiny\url{https://colab.research.google.com/drive/1RrlLMSom2M0E3iQk-XbDm66pDxtGAVtR}}
%%%%%%%%%%%%%%%%%%%%%%%%%%%%
\begin{document}
\maketitle
\setstretch{1.4}
\frame{\frametitle{Convergenza} Una successione di valori
$x_0,x_1,x_2,\ldots$ converge verso un valore $x^*$ se la distanza/errore
$\epsilon_k := \mid x_k-x\mid$ tende sempre pi\`u ad avvicinarsi a zero con il
crescere di $k$
$$\lim_{k\to+\infty} \epsilon_k = 0$$
Un algoritmo iterativo per la ricerca degli zeri di
una funzione genera una successione di valori $x_0,x_1,x_2,\ldots$,
tale che, nei casi in cui converge, tende verso il valore $x^*$ di uno
zero}

\frame{\frametitle{Ordine di convergenza}
Un algoritmo
iterativo ha ordine di convergenza $p$ se esistono due numeri $C\geq
0$ e $p\geq 0$ tali che $$\lim_{k\rightarrow\infty}
\frac{e_{k+1}}{e_k^p}=C,$$ ovvero $|x_{k+1}-x^*|<C |x_k-x^*|^p$
\vskip 2mm
Se $p=1$ si dice che l'ordine di convergenza
\`e lineare\\ superlineare con $1<p<2$, quadratico con $p=2$}

\frame{\frametitle{Convergenza dei vari algoritmi}
\begin{itemize}
\item L'algoritmo della bisezione e le sue varianti convergono linearmente ($p=1$)
\item L'algoritmo della secante converge superlinearmente\\
($p=\frac{1+\sqrt{5}}{2} \simeq 1.618$)
\item L'algoritmo della tangente converge quadraticamente ($p=2$)
\end{itemize}}


\frame{\frametitle{Convergenza su punti a tangenza orizzontale}
\begin{itemize}
\item Se $f'(x^*)=0$ (zero con tangente orizzontale), l'algoritmo della tangente ``rallenta'' e la convergenza \`e lineare e non quadratica
\item Es. con $f(x)=x^2$, $f(x)=2x$ allora $x^*=0$ e $f'(x^*)=0$.
\end{itemize}
\vskip 1mm
\centering
	\small
\begin{tabular}{cccc}
$k$&$x_k$&$f(x_k)$&$f'(x_k)$\\
	\hline
$0$	& $1$	&$1$&$2$\\
$1$	& $1-\frac{1}{2}=\frac{1}{2}$&$\frac{1}{4}$&$1$\\
$2$	&$\frac{1}{2}-\frac{\frac{1}{4}}{1}=\frac{1}{4}$&$\frac{1}{16}$&$\frac{1}{2}$\\
$3$	&$\frac{1}{4}-\frac{\frac{1}{16}}{\frac{1}{2}}=\frac{1}{8}$\\

\end{tabular}
\vskip 1mm	\emph{Ogni iterazione dimezza l'errore, convergenza lineare!}
\vskip 1mm la stessa cosa succede con l'algoritmo della secante
}

\end{document}
