\documentclass[professionalfonts]{beamer}
\newif\ifita
\itatrue % comment out to hide answers
%\itafalse
\usepackage[familydefault,light]{Chivo} 
\usepackage[T1]{fontenc}
\usenavigationsymbolstemplate{}
\usepackage[]{hyperref}
\usepackage{tikz,pgf,pgfarrows,pgfnodes,pgfbaseimage}
\graphicspath{{./Pics/}}
\usetikzlibrary{shapes}
\usepackage{setspace}
\newcommand{\evi}[1]{{\colorbox{yellow!50}{{#1}}}}
\newcommand{\exe}[1]{{\color{black!50}{{#1}}}}
\newcommand{\kw}[1]{{\colorbox{black!30}{\color{white}{#1}}}}
\tikzstyle{nd}=[circle,draw=black,thick,minimum size=.8cm,inner sep=1pt]
\setbeamercovered{transparent}
\usetheme{Singapore}
\tikzstyle{nodo}=[ellipse,draw=black!60,fill=black!10,line width=.7pt,minimum width=.7cm,minimum height=.4cm]
\usecolortheme[named=gray]{structure}
\setbeamercolor{block title}{bg=black!20,fg=black}
\setbeamercolor{block body}{bg=black!10,fg=black}

\usepackage{comment}
%%%%%%%%%%%%%%%%%%%%%
\ifita
\title{Algoritmi Numerici (Parte IV)}
\subtitle{[Lezione 4] Interpolazione Trigonometrica}
\else
\title{Numerics (Part IV)}
\subtitle{[Lectures 2 \& 3] Sp-Line Interpolation\\(Quadratic and Cubic)}
\fi
\date{}
\author{Alessandro Antonucci\\{\tt alessandro.antonucci@supsi.ch}}
%%%%%%%%%%%%%%%%%%%%%%%%%%%%
\begin{document}
    \maketitle
    \setbeamercovered{}
    %\setstretch{1.3}
    \frame{\frametitle{\ifita Interpolazione di Segnali Periodici \else Sp-line Approach \fi}
\begin{itemize}
\item Se i punti che devo interpolare rappresentano un fenomeno periodico
\item Allora conviene interpolarli con funzioni di tipo periodico
\item Al posto di una combinazione di potenze di $x$, uso una combinazione di seni e coseni di multipli di $x$
\item Ricordo che $\sin(jx)$ e $\cos(jx)$ sono funzioni periodiche di periodo $2\pi / j$ e quindi anche di periodo $2\pi$.
\end{itemize}
}
\frame{\frametitle{Interpolazione trigonometrica}
\begin{itemize}
\item Interpolo $n$ punti di coordinate $\{(x_i,y_i)\}_{i=0}^{n-1}$ ($n$ dispari)
\item Uso la funzione $f(x):= C + \sum_{j=1}^m \left[ a_j \cos (jx) + b_j \sin (jx) \right]$
\item Per bloccare i parametri $n=2m+1$, quindi $m=\frac{n-1}{2}$
\item Se i punti di appoggio dividono l'intervallo $[0,2\pi]$ in $n$ parti uguali, allora valgono le seguenti formule: 
$$a_j = \frac{2}{n} \sum_{i=0}^{n-1} \left[ y_i \cos(j x_i) \right], \quad \forall j=0,1,\ldots,m$$
$$b_j = \frac{2}{n} \sum_{i=0}^{n-1} \left[ y_i \sin(j x_i) \right], \quad \forall j=1,\ldots,m,$$
e $C=\frac{a_0}{2}$\\ ($C$ = media aritmetica delle $y$ dei punti di appoggio)
\end{itemize}}
\end{document}