\documentclass[professionalfonts]{beamer}
\newif\ifita
\itatrue % comment out to hide answers
\itafalse
\usepackage[familydefault,light]{Chivo} 
\usepackage[T1]{fontenc}
\usenavigationsymbolstemplate{}
\usepackage[]{hyperref}
\usepackage{tikz,pgf,pgfarrows,pgfnodes,pgfbaseimage}
\graphicspath{{./Pics/}}
\usetikzlibrary{shapes}
\usepackage{setspace}
\newcommand{\evi}[1]{{\colorbox{yellow!50}{{#1}}}}
\newcommand{\exe}[1]{{\color{black!50}{{#1}}}}
\newcommand{\kw}[1]{{\colorbox{black!30}{\color{white}{#1}}}}
\tikzstyle{nd}=[circle,draw=black,thick,minimum size=.8cm,inner sep=1pt]
\setbeamercovered{transparent}
\usetheme{Singapore}
\tikzstyle{nodo}=[ellipse,draw=black!60,fill=black!10,line width=.7pt,minimum width=.7cm,minimum height=.4cm]
\usecolortheme[named=gray]{structure}
\setbeamercolor{block title}{bg=black!20,fg=black}
\setbeamercolor{block body}{bg=black!10,fg=black}

%%%%%%%%%%%%%%%%%%%%%
\ifita
\title{Algoritmi Numerici (Parte III)}
\subtitle{[Lezione 3] Scomposizione LR ed Inversione Matrici}
\else
\title{Numerics (Part III)}
\subtitle{[Lecture 3] LR Decomposition and Matrix Inversion}
\fi
\date{}
\author{Alessandro Antonucci\\{\tt alessandro.antonucci@supsi.ch}}
\begin{document}
\maketitle
\setstretch{1.4}
%%%%%%%%%%%%%%%%%%%%%%%%%%%%
\frame{\frametitle{\ifita L'idea (caso scalare)\else Idea (scalar case) \fi}
$35 \cdot x = 8$ \\  %\quad Equazione Lineare\\
$7 \cdot 5 x = 8$\\ %\quad Scompongo il coefficiente\\
\vskip 2mm
$5 x = y$ \\%\quad Introduco una nuova variabile\\
$ 7 y = 8$ %\quad Riscrivo l'equazione con la nuova variabile
\vskip 2mm
$\begin{cases} 5x = y\\ 7y=8 \end{cases}$,\\ $y=\frac{8}{7}$,\\  $x=\frac{y}{5}=\frac{8}{35}$

\begin{center}
%Utile? No, nel caso scalare, a volte nel caso vettoriale
\end{center}
}

\frame{\frametitle{\ifita Scomposizione LR\else LR Decomposition \fi}
\begin{itemize}
\item \ifita Dato sistema lineare \else Given linear system \fi $\hat{A} \cdot \vec{x} = \vec{b}$
\item \ifita Scomposizione \else Decompose \fi $\hat{A}=\hat{L} \cdot \hat{R}$ \\
\ifita
($\hat{L}$ triangolare inferiore e $\hat{R}$ triangolare superiore)
\else
($\hat{L}$ lower triangular and $\hat{R}$ upper triangular)
\fi
\item 
\ifita Il sistema diventa \else The system rewrites as \fi
$\hat{L} \cdot \hat{R} \cdot \vec{x} = \vec{b}$
\item \ifita Equivale a coppia sistemi triangolari \else Equivalent to a pair of triangular systems\fi
$\left\{ \begin{array}{l} \hat{R} \cdot \vec{x} = \vec{y} \\ \hat{L} \cdot \vec{y} = \vec{b} \end{array} \right.$ 
\ifita
\item Entrambi i sistemi sono triangolari!
\item Risolviamo il secondo e poi il primo per sostituzione
\item Problema: come scomporre $\hat{A}$?
\else
\item Both systems are triangular!
\item Solve the second and then the first (both by substitution)
\item Problem: how to decompose $\hat{A}$ in the product $\hat{L}\cdot \hat{R}$?
\fi
\end{itemize}}

\frame{\frametitle{\ifita Scomposizione LR (ii) \else LR Decomposition (ii)\fi}
\begin{itemize}
\item \ifita Per scomporre $\hat{A}$ nel prodotto di due matrici triangolari basta fare Gauss (ignorando la colonna dei termini noti) \else
For $\hat{L}\hat{R}$ decomposition just do Gauss on $\hat{A}$ only (not $\hat{b}$) \fi 
\item \ifita La matrice $\hat{R}$ \`e quella restituita dall'algoritmo di Gauss \else
Matrix $\hat{R}$ is the output of Gauss \fi
\item \ifita La matrice $\hat{L}$ ha \else Matrix $\hat{L}$ is such that\fi
\begin{itemize} 
\item \ifita sopra la diagonale: tutti zeri \else all zeros over the diagonal \fi
\item \ifita sulla diagonale: tutti uno \else all ones on the diagonal \fi
\item \ifita sotto la diagonale: valori coeff per annichilire elementi $\hat{A}$ 
\else coeffs used to annihilate $\hat{A}$ elements\fi
\end{itemize}
\end{itemize}}

\frame{\frametitle{\ifita Esempio $2\times 2$ (Scomposizione) \else $2\times 2$ Example (Decomposition) \fi}
\begin{itemize}
\item $\hat{A} = \left[ \begin{array}{cc} 2&1 \\ -1&-2 \end{array}  \right]$
\vskip 4mm
%,  $\vec{b} = \left[ \begin{array}{c} 7 \\ -5 \end{array}  \right]$
\item Gauss \ifita solo su \else only on \fi $\vec{A}$, {\tt pivot}$=2$, {\tt coeff}$=-\frac{1}{2}$
\vskip 4mm
\item 
$\hat{L} = \left[ \begin{array}{cc} 1&0 \\ -\frac{1}{2}&1 \end{array}  \right]$ , $\hat{R} = \left[ \begin{array}{cc} 2&1 \\ 0&-\frac{3}{2} \end{array}  \right]$
\vskip 8mm
\item \ifita Esercizio (verifica) \else Exercise (check the decomposition) \fi \vskip 3mm
$\left[ \begin{array}{cc} 1&0 \\ -\frac{1}{2}&1 \end{array}  \right]
\cdot 
\left[ \begin{array}{cc} 2&1 \\ 0&-\frac{3}{2} \end{array}  \right]
=
\left[ \begin{array}{cc} 2&1 \\ -1&-2 \end{array}  \right]$
\end{itemize}
%$\hat{A} = \left[ \begin{array}{cc} \color{red}{2}&1 \\ -1&-2 \end{array}  \right]$
}

\frame{\frametitle{\ifita Esempio $2\times 2$ (Soluzione) \else $2\times 2$ Example (Solution) \fi}
\begin{itemize}
\item $\hat{A} = \left[ \begin{array}{cc} 2&1 \\ -1&-2 \end{array}  \right]$, $\vec{b} = \left[ \begin{array}{c} 7 \\ -5 \end{array}  \right]$
\item \ifita Risolvo prima \else Solving first \fi $\hat{L}\cdot \vec{y} = \vec{b}$ \ifita e poi \else and then \fi $\hat{R} \cdot \vec{x} = \vec{y}$.
\end{itemize}
$$\left[ \begin{array}{cc} 1&0 \\ -\frac{1}{2}&1 \end{array}  \right]
\cdot
\left[ \begin{array}{c} \color{red}{y_1} \\ \color{red}{y_2} \end{array}  \right]
=
\left[ \begin{array}{c} 7 \\ -5 \end{array}  \right]
$$
$$\left[ \begin{array}{cc} 2&1 \\ 0&-\frac{3}{2} \end{array}  \right]
\cdot
\left[ \begin{array}{c} \color{blue}{x_1} \\ \color{blue}{x_2} \end{array}  \right]
=
\left[ \begin{array}{c} \color{red}{7} \\ \color{red}{-\frac{3}{2}} \end{array}  \right]
$$
\begin{itemize}
\item $\color{blue}{x_2}=1$, $\color{blue}{x_1}=3$
\end{itemize}
}

\frame{\frametitle{\ifita Analisi di Complessit\`a\else Complexity Analysis \fi}
\begin{itemize}
\ifita
\item LR ha stessa complessit\`a di Gauss $O(n^3)$
\item Devo poi fare due sostituzioni $2 O(n^2)$
\item Stessa complessit\`a di Gauss + sostituzione
\item Nessun vantaggio se bisogna risolvere un solo sistema
\item Vantaggi se bisogna risolvere tanti sistemi,\\se tutti con la stessa matrice dei coefficienti
\else
\item LR complexity same as Gauss $O(n^3)$
\item After LR, two substitutions are needed $2 O(n^2)$
\item Same complexity as Gauss + substitution
\item No benefits when solving a singly system
\item Benefits when solving many systems,\\all with the same coefficients matrix
\fi
\end{itemize}
}

\frame{\frametitle{\ifita Inversione Matrici (Applicazione) \else Matrix Inversion (Application)\fi}
\begin{itemize}
\item $\hat{A} = \left[ \begin{array}{cc} 2&1 \\ -1&-2 \end{array}  \right]$ \quad $\hat{A}^{-1} = \left[ 
\begin{array}{cc} \color{red}{\bullet}  & \color{blue}{\bullet} \\ 
\color{red}{\bullet} & \color{blue}{\bullet} \end{array} 
\right]$
\vskip 2mm
\item$\hat{A}\cdot \hat{A}^{-1}=
\left[ \begin{array}{cc} 2&1 \\ -1&-2 \end{array}  \right]
\cdot 
\left[ \begin{array}{cc} \color{red}{\bullet}  & \color{blue}{\bullet} \\ 
\color{red}{\bullet} & \color{blue}{\bullet} \end{array} 
\right]
=
\left[ \begin{array}{cc} 1&0 \\ 0&1 \end{array}  \right]=\hat{I}$
\item \ifita Risolvo (con LR) due sistemi con stessa matrice 
\else Solving (by LR) two systems with the same
$\hat{A}$:\fi
\vskip 2mm
\begin{itemize}
\item $\left[ \begin{array}{cc} 2&1 \\ -1&-2 \end{array}  \right]
\cdot 
\left[ \begin{array}{cc} \color{red}{\bullet}   \\ 
\color{red}{\bullet}  \end{array} 
\right]
=
\left[ \begin{array}{cc} \color{red}{1} \\ \color{red}{0} \end{array}  \right]
$
\vskip 2mm
\item $\left[ \begin{array}{cc} 2&1 \\ -1&-2 \end{array}  \right]
\cdot 
\left[ \begin{array}{cc} \color{blue}{\bullet}   \\ 
\color{blue}{\bullet}  \end{array} 
\right]=\left[ \begin{array}{cc} \color{blue}{0} \\ \color{blue}{1} \end{array}  \right]$
\end{itemize}
\end{itemize}}
\end{document}
