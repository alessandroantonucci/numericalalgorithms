\documentclass[professionalfonts]{beamer}
\usepackage[familydefault,light]{Chivo} 
\usepackage[T1]{fontenc}
\usenavigationsymbolstemplate{}
\usepackage[]{hyperref}
\usepackage{tikz,pgf,pgfarrows,pgfnodes,pgfbaseimage}
\graphicspath{{./Pics/}}
\usetikzlibrary{shapes}
\usepackage{setspace}
\newcommand{\evi}[1]{{\colorbox{yellow!50}{{#1}}}}
\newcommand{\exe}[1]{{\color{black!50}{{#1}}}}
\newcommand{\kw}[1]{{\colorbox{black!30}{\color{white}{#1}}}}
\tikzstyle{nd}=[circle,draw=black,thick,minimum size=.8cm,inner sep=1pt]
\setbeamercovered{transparent}
\usetheme{Singapore}
\tikzstyle{nodo}=[ellipse,draw=black!60,fill=black!10,line width=.7pt,minimum width=.7cm,minimum height=.4cm]
\usecolortheme[named=gray]{structure}
\setbeamercolor{block title}{bg=black!20,fg=black}
\setbeamercolor{block body}{bg=black!10,fg=black}

\title{Algoritmi Numerici (Parte II)}
\subtitle{[Lezione 1] Algoritmo di bisezione e regula falsi}
\author{Alessandro Antonucci\\{\tt alessandro.antonucci@supsi.ch}}
\date{\tiny\url{https://colab.research.google.com/drive/1l5_Nsbk2LIETfD3IDm74w6Wl0HhrjRcB}}
\begin{document}
\maketitle
\setstretch{1.4}
\frame{\frametitle{Bracketing}
\begin{itemize}
\item Data una funzione $f$ (continua)
\item \evi{Se} sull'intervallo $[a,b]\subset \mathbb{R}$,\\la funzione cambia segno,\\ovvero $f(a)\cdot f(b) < 0$
\item \evi{Allora} la funzione ha uno zero\\in questo intervallo,\\ovvero $\exists$ $x^* \in [a,b]$ tale che $f(x^*)=0$
\end{itemize}}
%%%%%%%%%%%%%%%%%%%%%%%%%%%%
\frame{\frametitle{La bisezione}
\begin{itemize}
\item Dopo aver localizzato $x^*$ sull'intervallo $[a,b]$
\item Prendo il punto medio $c:=\frac{a+b}{2}$
\item La funzione cambia segno su $[a,c]$ \evi{oppure} su $[c,b]$
\item Nel primo caso $x^*\in[a,c]$, \\ nel secondo $x^*\in [c,b]$
\end{itemize}
\begin{center}
\color{black!50}{In entrambi i casi, il nuovo intervallo \`e la met\`a del vecchio}
\end{center}}
%%%%%%%%%%%%%%%%%%%%%%%%%%%%
\frame{\frametitle{La bisezione (pseudo codice)}
\begin{columns}
\begin{column}[T]{0.4\textwidth}
\tt
if f(a)*f(b)<0: \\
\quad for k in range(n): \\      
\quad\quad	c = (a+b)/2;  \\  	
\quad\quad	if f(a)*f(c) < 0:	\\
\quad\quad\quad b = c\\     
\quad\quad else\\
\quad\quad\quad a = c\\
\end{column}
\begin{column}[T]{0.4\textwidth}
L'algoritmo viene iterato
\begin{center}
$n$ (numero fissato) volte
\vskip 2mm
oppure
\vskip 2mm
fino a quando\\
$|b-a|$ diventa piccolo
\end{center}
\end{column}
\end{columns}}
%%%%%%%%%%%%%%%%%%%%%%%%%%%%
\frame{\frametitle{Promemoria}
\begin{itemize} 
\item Dato l'intervallo $[a,b]$:
	\begin{itemize}
	\item la sua ampiezza \`e $b-a$
	\item il suo punto medio  $\frac{a+b}{2}$
	\end{itemize}
\item Es. l'intervallo $[4,10]$ ha ampiezza $6$ e punto medio $x=7$
\item Se $x^*\in[a,b]$, la stima puntuale \`e il punto medio $c:=\frac{a+b}{2}$
\item L'errore peggiore \`e $\epsilon_{worst}=\frac{b-a}{2}$
\item Ogni altra stima produrrebbe un errore superiore
\end{itemize}
}
	
	
\frame{\frametitle{Analisi precisione}
\begin{itemize}
\item Intervallo iniziale $[a^0,b^0]$, stima puntuale $c^0:=\frac{a^0+b^0}{2}$
\item $x^* \in [a^0,b^0]$ e $\epsilon^0:=|x^*-c^0|< \frac{b^0-a^0}{2}$
\item Analogamente, $\epsilon^k < \frac{b^0-a^0}{2^{k+1}}$
\end{itemize}
\begin{center}
\color{black!50}{Con la bisezione posso quindi prevedere\\quante iterazioni servono per rendere l'errore minore\\di un valore prefissato}
\end{center}}

\frame{\frametitle{Osservazione}
\begin{itemize}
\item L'algoritmo di bisezione si basa sulla scelta di un punto $c$ interno all'intevallo $[a,b]$
\item Scegliere il punto medio ha il vantaggio/svantaggio di rendere il nuovo intervallo la met\`a di quello vecchio
\item Ogni altra scelta di $c\in[a,b]$ permette comunque di procedere
\item In particolare, nella scelta di $c$ pu\`o essere utile considerare il valore (e non solo il segno) della
	funzione in $f(a)$ e $f(b)$
\end{itemize}
}

\frame{\frametitle{Regula Falsi}
\begin{itemize}
\item Variante dell'algoritmo di bisezione 
\item $c$ \`e il punto d'incontro con l'asse $x$ della retta che passa per i punti di coordinate $(a,f(a))$ e $(b,f(b))$
\end{itemize}
\vskip 2mm
$$c:= a - \frac{f(a)}{ \frac{f(b)-f(a)}{b-a}}$$}

\frame{\frametitle{Regula falsi}
\begin{tikzpicture}[domain=0:4,scale=1.2]
\draw[->,thick] (-1,0) -- (5,0);
\draw[densely dotted,black!60!green,thick] (0,3) -- (4,-7/3);
\draw[black!60!green,thick] (9/4,-.1) -- (9/4,.1) node[above] {$c$};
\draw[red,thick,densely dotted] (4,0) -- (4,-7/3);
\draw [red,fill] (4,-7/3) circle [radius=0.05] node[right] {$(b,f(b))$};
\draw[red,thick] (4,-.1) -- (4,.1) node[above] {$b$};
\draw[blue,densely dotted,thick] (0,0) -- (0,3);
\draw [blue,fill] (0,3) circle [radius=0.05] node[left] {$(a,f(a))$};
\draw[blue,thick] (0,.1) -- (0,-.1) node[below] {$a$};
\draw[thick,color=orange] plot[id=exp] function{3-x*x*1/3}; 
\draw[black!60!green] (4,2) node[] {$c=a-\frac{f(a)}{\frac{f(b)-f(a)}{b-a}}$};
%\draw [fill] (9,5) circle [radius=0.05];
%\draw[very thin,color=gray] (-0.1,-1.1) grid (3.9,3.9);
% node[right]{} {$x%$};
%\draw[->] (0,-3.2) -- (0,6.2) node[above] {$f(x)$};
%\draw[color=red] plot[id=x] function{x} 
%node[right] {$f(x) =x$};
%\draw[color=blue] plot[id=sin] function{sin(x)} 
%node[right] {$f(x) = \sin x$};
%\draw[color=blue] (0,-3) node[left] {$(a,f(a))$};
%\draw[color=blue] (3,6) node[left] {$(b,f(b))$};
%\draw[color=blue] (0,0) node[left] {$a$};
%\draw[color=blue] (1.2,0) node[left] {$c$};
%\draw[color=blue] (3,0) node[left] {$b$};
%\draw (3,6)--(0,-3);
%\draw[color=orange] plot[id=exp] function{x*x-3} 
%node[right] {$f(x) = \mathrm x^2-3$};
\end{tikzpicture}
}


\frame{\frametitle{Regula Falsi (ii)}
\begin{itemize}
\item Tipicamente, dopo un certo numero di iterazioni, la regula falsi sposta sempre l'estremo destro (o sempre quello sinistro) dell'intervallo
\item Se sull'intervallo la funzione \`e sempre concava (o convessa), la retta che congiunge i due punti estremi \`e sempre a sx (o sempre a dx) dello zero della funzione
\item In pratica, l'ampiezza dell'intervallo non tende a zero, ma il punto $c$ tende a $x^*$
\end{itemize}}


\frame{\frametitle{Ibrido Regula Falsi + Bisezione}
	\begin{itemize}
		\item Dato $[a,b]$ (su cui $f$ cambia segno)
		\item $[a,b]=RegFals(a,b)$ (una volta RF)
		\item WHILE($\ldots$ )$\{[a,b]=Bisez(a,b)\}$ \\
			(itero bisez finch\'e non si muove l'estremo fisso della RF)
		\item Ricomincio con RF
	\end{itemize}
	\vskip 2mm
	\begin{center}
		\emph{Supera problema estremo fisso della RF}
\end{center}}


\end{document}
