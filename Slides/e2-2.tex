\documentclass[professionalfonts]{beamer}
\newif\ifanswers
%\answerstrue % comment out to hide answers
\answersfalse
\usepackage[familydefault,light]{Chivo} 
\usepackage[T1]{fontenc}
\usenavigationsymbolstemplate{}
\usepackage[]{hyperref}
\usepackage{tikz,pgf,pgfarrows,pgfnodes,pgfbaseimage}
\graphicspath{{./Pics/}}
\usetikzlibrary{shapes}
\usepackage{setspace}
\newcommand{\evi}[1]{{\colorbox{yellow!50}{{#1}}}}
\newcommand{\exe}[1]{{\color{black!50}{{#1}}}}
\newcommand{\kw}[1]{{\colorbox{black!30}{\color{white}{#1}}}}
\tikzstyle{nd}=[circle,draw=black,thick,minimum size=.8cm,inner sep=1pt]
\setbeamercovered{transparent}
\usetheme{Singapore}
\tikzstyle{nodo}=[ellipse,draw=black!60,fill=black!10,line width=.7pt,minimum width=.7cm,minimum height=.4cm]
\usecolortheme[named=gray]{structure}
\setbeamercolor{block title}{bg=black!20,fg=black}
\setbeamercolor{block body}{bg=black!10,fg=black}

\ifanswers
\title{Algoritmi Numerici (Parte II)}
\subtitle{[Esercitazione 1] Algoritmo di Bisezione}
\else
\title{Numerics (Part II)}
\subtitle{[Exercise Session 1] Regula Falsi Algorithm}
\fi
\date{}
\author{Alessandro Antonucci\\{\tt alessandro.antonucci@supsi.ch}}
\begin{document}
\maketitle
\setstretch{1.4}
\frame{\frametitle{\ifanswers Esercizio \else Exercise \fi}
\begin{enumerate}
\item In Excel, find a zero of $f(x)=1/x-1$ in interval $[\frac{2}{3},3]$\\by three iterations of the (pure) regula falsi algorithm and comment on the main issue with this approach
\item With the same function and the same starting interval estimate the zero by two iteration of the hybrid algorithm Regula Falsi + bisection
\item Find the exact zero of $f$ in order to decide whether the pure bisection would have been more accurate
\item Do the same in Python and upload your code (Assign. 2)
\end{enumerate}}
\end{document}


\item \ifanswers Data una funzione $f$ (continua)\else Given a (continuous) function $f$ \fi
\ifanswers
\item \evi{Se} sull'intervallo $[a,b]\subset \mathbb{R}$,\\la funzione cambia segno,\\ovvero $f(a)\cdot f(b) < 0$
\else
\item \evi{If} in the interval $[a,b]\subset \mathbb{R}$,\\function $f$ changes its sign,\\that means $f(a)\cdot f(b) < 0$
\fi
\ifanswers
\item \evi{Allora} la funzione ha uno zero\\in questo intervallo,\\ovvero $\exists$ $x^* \in [a,b]$ tale che $f(x^*)=0$
\else
\item \evi{Then} the function has a zero\\in this interval,\\that means $\exists$ $x^* \in [a,b]$ such that $f(x^*)=0$
\fi
\end{itemize}}
%%%%%%%%%%%%%%%%%%%%%%%%%%%%
\frame{\frametitle{\ifanswers La bisezione\else Bisection Algorithm\fi}
\begin{itemize}
\item \ifanswers Dopo aver localizzato $x^*$ sull'intervallo $[a,b]$ \else
After having localized $x^*$ in the interval $[a,b]$ \fi
\item \ifanswers Prendo il punto medio \else Take the midpoint \fi $c:=\frac{a+b}{2}$
\ifanswers
\item La funzione cambia segno su $[a,c]$ \evi{oppure} su $[c,b]$
\item Nel primo caso $x^*\in[a,c]$, \\ nel secondo $x^*\in [c,b]$
\else
\item The function changes its sign in $[a,c]$ \evi{or} in $[c,b]$
\item In the first case $x^*\in[a,c]$, \\ in the second $x^*\in [c,b]$
\fi
\end{itemize}
\begin{center}
\ifanswers
\color{black!50}{In entrambi i casi, il nuovo intervallo \`e la met\`a del vecchio}
\else
\color{black!50}{In both cases, the size of the new interval\\is half of the previous one}
\fi
\end{center}}
%%%%%%%%%%%%%%%%%%%%%%%%%%%%
\frame{\frametitle{\ifanswers La bisezione (pseudo codice)\else Bisection (pseudocode) \fi}
\begin{columns}
\begin{column}[T]{0.4\textwidth}
\tt
if f(a)*f(b)<0: \\
\quad for k in range(n): \\      
\quad\quad	c = (a+b)/2;  \\  	
\quad\quad	if f(a)*f(c) < 0:	\\
\quad\quad\quad b = c\\     
\quad\quad else\\
\quad\quad\quad a = c\\
\end{column}
\begin{column}[T]{0.4\textwidth}
\ifanswers
L'algoritmo viene iterato
\else
Algorithm iterated
\fi
\begin{center}
\ifanswers
$n$ (numero fissato) volte
\else
$n$ (fixed number) times
\fi
\vskip 2mm
\ifanswers oppure \else or \fi
\vskip 2mm
\ifanswers
fino a quando\\
$|b-a|$ diventa piccolo
\else
until $|b-a|$\\becomes small
\fi
\end{center}
\end{column}
\end{columns}}
%%%%%%%%%%%%%%%%%%%%%%%%%%%%
\frame{\frametitle{\ifanswers Promemoria \else Don't forget\fi}
\begin{itemize} 
\item \ifanswers Dato l'intervallo \else Given the interval \fi $[a,b]$:
	\begin{itemize}
	\item \ifanswers la sua ampiezza \`e \else its size is \fi$b-a$
	\item \ifanswers il suo punto medio \else its midpoint is \fi $\frac{a+b}{2}$
	\end{itemize}
\ifanswers
\item Es. l'intervallo $[4,10]$ ha ampiezza $6$ e punto medio $x=7$
\item Se $x^*\in[a,b]$, la stima puntuale \`e il punto medio $c:=\frac{a+b}{2}$
\item L'errore peggiore \`e $\epsilon_{worst}=\frac{b-a}{2}$
\item Ogni altra stima produrrebbe un errore superiore
\else
\item Ex. interval $[4,10]$ has size $6$ and midpoint $x=7$
\item If $x^*\in[a,b]$, the best estimate is the midpoint $c:=\frac{a+b}{2}$
\item We pessimistic error is $\epsilon_{worst}=\frac{b-a}{2}$
\item Any other estimate not in the midpoint is less accurate
\fi
\end{itemize}
}
	
	
\frame{\frametitle{\ifanswers Analisi precisione \else Precision\fi}
\begin{itemize}
\item \ifanswers Intervallo iniziale \else Initial interval \fi $[a^0,b^0]$, 
\ifanswers stima puntuale \else point estimate \fi $c^0:=\frac{a^0+b^0}{2}$
\item $x^* \in [a^0,b^0]$ e $\epsilon^0:=|x^*-c^0|< \frac{b^0-a^0}{2}$
\item \ifanswers Analogamente \else Similarly \fi, $\epsilon^k < \frac{b^0-a^0}{2^{k+1}}$
\end{itemize}
\begin{center}
\ifanswers
\color{black!50}{Con la bisezione posso quindi prevedere\\quante iterazioni servono per rendere l'errore minore\\di un valore prefissato}
\else
\color{black!50}{With bisection it is possible to predict how many iterations are needed to make the error smaller than a fixed value}
\fi
\end{center}}

\frame{\frametitle{\ifanswers Osservazione \else Remark \fi}
\begin{itemize}
\ifanswers
\item L'algoritmo di bisezione si basa sulla scelta di un punto $c$ interno all'intervallo $[a,b]$
\item Scegliere il punto medio ha il vantaggio/svantaggio di rendere il nuovo intervallo la met\`a di quello vecchio
\item Ogni altra scelta di $c\in[a,b]$ permette comunque di procedere
\item In particolare, nella scelta di $c$ pu\`o essere utile considerare il valore (e non solo il segno) della funzione in $f(a)$ e $f(b)$
\else
\item Bisection algorithm needs to pick a point $c$ inside the interval $[a,b]$
\item Pick the midpoint has the advantage/disadvantage of making the new interval half of the previous one
\item Any other $c\in[a,b]$ allow to run the algorithm in a similar way
\item When choosing  $c$ it might be important the value (and not only the sign) of the function in $f(a)$ and $f(b)$
\fi
\end{itemize}
}

\frame{\frametitle{Regula Falsi}
\begin{itemize}
\item \ifanswers Variante dell'algoritmo di bisezione \else A variant of bisection \fi 
\ifanswers
\item $c$ \`e il punto d'incontro con l'asse $x$ della retta che passa per i punti di coordinate $(a,f(a))$ e $(b,f(b))$
\else
\item $c$ is the point where the $x$ axis cut the segment connecting the points of coordinates $(a,f(a))$ e $(b,f(b))$

\fi
\end{itemize}
\vskip 2mm
$$c:= a - \frac{f(a)}{ \frac{f(b)-f(a)}{b-a}}$$}

\frame{\frametitle{Regula falsi}
\begin{tikzpicture}[domain=0:4,scale=1.2]
\draw[->,thick] (-1,0) -- (5,0);
\draw[densely dotted,black!60!green,thick] (0,3) -- (4,-7/3);
\draw[black!60!green,thick] (9/4,-.1) -- (9/4,.1) node[above] {$c$};
\draw[red,thick,densely dotted] (4,0) -- (4,-7/3);
\draw [red,fill] (4,-7/3) circle [radius=0.05] node[right] {$(b,f(b))$};
\draw[red,thick] (4,-.1) -- (4,.1) node[above] {$b$};
\draw[blue,densely dotted,thick] (0,0) -- (0,3);
\draw [blue,fill] (0,3) circle [radius=0.05] node[left] {$(a,f(a))$};
\draw[blue,thick] (0,.1) -- (0,-.1) node[below] {$a$};
\draw[thick,color=orange] plot[id=exp] function{3-x*x*1/3}; 
\draw[black!60!green] (4,2) node[] {$c=a-\frac{f(a)}{\frac{f(b)-f(a)}{b-a}}$};
%\draw [fill] (9,5) circle [radius=0.05];
%\draw[very thin,color=gray] (-0.1,-1.1) grid (3.9,3.9);
% node[right]{} {$x%$};
%\draw[->] (0,-3.2) -- (0,6.2) node[above] {$f(x)$};
%\draw[color=red] plot[id=x] function{x} 
%node[right] {$f(x) =x$};
%\draw[color=blue] plot[id=sin] function{sin(x)} 
%node[right] {$f(x) = \sin x$};
%\draw[color=blue] (0,-3) node[left] {$(a,f(a))$};
%\draw[color=blue] (3,6) node[left] {$(b,f(b))$};
%\draw[color=blue] (0,0) node[left] {$a$};
%\draw[color=blue] (1.2,0) node[left] {$c$};
%\draw[color=blue] (3,0) node[left] {$b$};
%\draw (3,6)--(0,-3);
%\draw[color=orange] plot[id=exp] function{x*x-3} 
%node[right] {$f(x) = \mathrm x^2-3$};
\end{tikzpicture}
}


\frame{\frametitle{Regula Falsi (ii)}
\begin{itemize}
\ifanswers
\item Tipicamente, dopo un certo numero di iterazioni, la regula falsi sposta sempre l'estremo destro (o sempre quello sinistro) dell'intervallo
\item Se sull'intervallo la funzione \`e sempre concava (o convessa), la retta che congiunge i due punti estremi \`e sempre a sx (o sempre a dx) dello zero della funzione
\item In pratica, l'ampiezza dell'intervallo non tende a zero, ma il punto $c$ tende a $x^*$
\else
\item Typically, after some iterations, regula falsi always move the right point $b$ (or the left point $a$) in the interval
\item If in the interval the function is concave (o convex), the segment connecting the two extreme points cuts the x-axis always to the left (or right) of the zero of the function
\item In practice, the size of the interval does not tend to zero (even if $c$ tends to $x^*$)
\fi
\end{itemize}}


\frame{\frametitle{\ifanswers Ibrido Regula Falsi + Bisezione \else Hybrid Regula Falsi + Bisection\fi}
	\begin{itemize}
\ifanswers
\item Dato $[a,b]$ (su cui $f$ cambia segno)
\item $[a,b]=RegFals(a,b)$ (una volta RF)
\item WHILE($\ldots$ )$\{[a,b]=Bisez(a,b)\}$ \\
(itero bisez finch\'e non si muove l'estremo fisso della RF)
\item Ricomincio con RF
\else
\item Given $[a,b]$ (such that $f$ changes signature)
\item $[a,b]=RegFals(a,b)$ (single RF iteration)
\item WHILE($\ldots$ )$\{[a,b]=Bisect(a,b)\}$ \\
(iterating bisection until the other extreme point moves)
\item Restart with RF
\fi
	\end{itemize}
	\vskip 2mm
\begin{center}
\ifanswers
\emph{Supera problema estremo fisso della RF}
\else
\emph{Solves the problem of the fixed extreme point with RF}
\fi
\end{center}}
\end{document}
