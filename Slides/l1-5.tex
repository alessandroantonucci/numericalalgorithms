\documentclass[professionalfonts]{beamer}
\newif\ifanswers
%\answerstrue % comment out to hide answers
\answersfalse
\usepackage[familydefault,light]{Chivo} 
\usepackage[T1]{fontenc}
\usenavigationsymbolstemplate{}
\usepackage[]{hyperref}
\usepackage{tikz,pgf,pgfarrows,pgfnodes,pgfbaseimage}
\graphicspath{{./Pics/}}
\usetikzlibrary{shapes}
\usepackage{setspace}
\newcommand{\evi}[1]{{\colorbox{yellow!50}{{#1}}}}
\newcommand{\exe}[1]{{\color{black!50}{{#1}}}}
\newcommand{\kw}[1]{{\colorbox{black!30}{\color{white}{#1}}}}
\tikzstyle{nd}=[circle,draw=black,thick,minimum size=.8cm,inner sep=1pt]
\setbeamercovered{transparent}
\usetheme{Singapore}
\tikzstyle{nodo}=[ellipse,draw=black!60,fill=black!10,line width=.7pt,minimum width=.7cm,minimum height=.4cm]
\usecolortheme[named=gray]{structure}
\setbeamercolor{block title}{bg=black!20,fg=black}
\setbeamercolor{block body}{bg=black!10,fg=black}

\ifanswers
\title{Algoritmi Numerici (Parte I)}
\subtitle{[Lezione 5] Operazioni Aritmetiche Bitwise}
\else
\title{Numerics (Part I)}
\subtitle{[Lecture 5] Bitwise Arithmetics}
\fi
\date{}
\author{Alessandro Antonucci\\{\tt alessandro.antonucci@supsi.ch}}
%%%%%%%%%%%%%%%%%%%%%%%%%%%%

%\documentclass[professionalfonts,handout]{beamer}
%\usepackage{circuitikz}
%\usepackage[familydefault,light]{Chivo} 
\usepackage[T1]{fontenc}
\usenavigationsymbolstemplate{}
\usepackage[]{hyperref}
\usepackage{tikz,pgf,pgfarrows,pgfnodes,pgfbaseimage}
\graphicspath{{./Pics/}}
\usetikzlibrary{shapes}
\usepackage{setspace}
\newcommand{\evi}[1]{{\colorbox{yellow!50}{{#1}}}}
\newcommand{\exe}[1]{{\color{black!50}{{#1}}}}
\newcommand{\kw}[1]{{\colorbox{black!30}{\color{white}{#1}}}}
\tikzstyle{nd}=[circle,draw=black,thick,minimum size=.8cm,inner sep=1pt]
\setbeamercovered{transparent}
\usetheme{Singapore}
\tikzstyle{nodo}=[ellipse,draw=black!60,fill=black!10,line width=.7pt,minimum width=.7cm,minimum height=.4cm]
\usecolortheme[named=gray]{structure}
\setbeamercolor{block title}{bg=black!20,fg=black}
\setbeamercolor{block body}{bg=black!10,fg=black}

%\title{Algoritmi Numerici (Parte I)}
%\subtitle{[Lezione 6] Operazioni aritmetiche}
%\author{Alessandro Antonucci\\{\tt alessandro.antonucci@supsi.ch}}
%\date{\tiny\url{https://colab.research.google.com/drive/10HxMjR7A0Qa8X3cRe2ZCgO-rtRyYvVKY}}
%%%%%%%%%%%%%%%%%%%%%%%%%%%%
\begin{document}
\maketitle
\frame{\frametitle{Somma}
\begin{itemize}
\item \ifanswers Sistema a 4 bit con complemento a 2 \else 4-bit systems with two's complement \fi
\item \ifanswers Somma con circuiti con porte Booleane \else Sum by Boolean gates \fi (\href{https://upload.wikimedia.org/wikipedia/commons/thumb/9/92/Halfadder.gif/220px-Halfadder.gif}{link}) \\ \exe{(ex. $0+0=1+1=0$ \quad $0+1=1+0=1$ XOR)} 
\item $0010 + 0011 = 0101$ \exe{$2+3=5$} 
\item $0010 + 1100 = 1110$ \exe{$2+(-4)=-2$}
\item $0011 + 0110 = 1001$ \exe{$3+6 \neq -7$} overflow!
\item $1100 + 1010 = 0110$ \exe{$-4-6 \neq 10$} underflow!
\end{itemize}
\begin{center}
\emph{ \evi{Over/under flow \ifanswers se il primo bit \`e uguale \else if the first bit is equal\fi}
\evi{\ifanswers nei due addendi, ma diverso nella somma\else in the two addends, but different in the sum\fi}}
\end{center}
\begin{itemize}
\item $1100+1100=1000$ \exe{$-4-4=-8$} \\(\ifanswers giusto anche con riporto\else correct even with the carry\fi)
\end{itemize}}

\frame{\frametitle{\ifanswers Sottrazione (e precisione multipla)\else Subtraction (and multiple precision) \fi}
\begin{itemize}
\item \ifanswers Somma di due numeri a 4 bit: \else Summing two numbers (4 bit): \fi
\begin{itemize}
\item \ifanswers Two addends \else Due addendi \fi (4+4 bit)
\item Carry (bit $C$) \ifanswers si attiva per il riporto \else active when bit carried over \fi \exe{$1+1=1$ AND}
\item Overflow (bit $V$) \ifanswers si attiva per identificare \else active to detect \fi over/under
\end{itemize}
\item \ifanswers Numeri pi\`u grandi \else Bigger numbers \fi
\begin{itemize}
\item \ifanswers Precisione multipla (tante locazioni affiancate) \else Multiple precisions (many bit memories side-by-side) \fi
\item \ifanswers Inizializzo \else Initialize \fi $C=0$  (\ifanswers non importa se alla fine \else no matter if eventually \fi $C=1$)
\item \ifanswers Risultato corretto solo se \else Result correct if \fi $V=0$
\end{itemize}
\item \ifanswers Sottrazione \else Subtraction \fi
\begin{itemize}
\item a - b = a + (-b)
\item \ifanswers Per ricavare \else To obtain \fi $-b$, $b$ in base 2 \\
\ifanswers (ricopio da dx finch\'e non trovo $1$, scrivo quello, poi nego)
\else (copying bits from the right, if $1$, copy it, then negate) \fi
\end{itemize}
\end{itemize}}

\frame{\frametitle{\ifanswers Moltiplicazione e Divisione \else Multiplication and Division \fi}
\begin{itemize}
\ifanswers
\item Moltiplicazioni in colonna = somme numeri spostati a sx
\item ASL (arithmetic shift left) sposto i bit a sx (uno zero a dx)
\item Es. ASR(0011)=0110 
\item \`E moltiplicazione per 2 (se $V=0$ vale anche con complemento a 2)
\item Divisione: analoga ma sposto a dx
\item ASR (arithmetic shift right), divisione (intera) per 2
\else
\item Bitwise Multiplications = summing numbers shifted to the left
\item ASL (arithmetic shift left) shifring numbers to the left (extra zero to the right)
\item Es. ASR(0011)=0110 
\item Multiplication by 2 (if $V=0$ valid also with two's complement)
\item Division: just the same but moving to the right
\fi
\end{itemize}}

\frame{\frametitle{\ifanswers Aritmetica Float\else Float Arithmetic\fi}
\begin{itemize}
\item \ifanswers Dati due numeri (float) \else Given two (float) numbers \fi
\begin{itemize}
\item $a = m_1 \cdot b^q$ 
\item	$b = m_2 \cdot b^p$ ($q>p$)
\end{itemize}
\item \ifanswers Posso calcolare \else I can compute \fi

\begin{itemize}
\item $a+b = (m_1 + m_2 \cdot b^{p-q}) \cdot b^q$
\item $a \cdot b = (m_1 \cdot m_2) \cdot b^{q+p}$
\item $a / b= (m_1 / m_2) \cdot b^{q-p}$
\end{itemize}
\ifanswers
\item Le operazioni sulla mantissa introducono un errore\\
(riducono le cifre significative)
\else
\item Operations on the mantissa might produce an additional error\\
(less significant digits)
\fi
\end{itemize}}

\frame{\frametitle{Under/over flow \ifanswers con numeri float\else with float numbers \fi} 
\ifanswers Formati float-like caratterizzati da:\else Float formats characterized by: \fi
\begin{itemize}
\item $m_1$ \ifanswers numero pi\`u piccolo rappresentabile \else smallest (machine) number\fi
\item $M_1$ \ifanswers numero pi\`u grande \else largest (machine) number \fi
\end{itemize}
\ifanswers Quando inserisco un numero $x$, il sistema restituisce \else When $x$ is the number in input, system returns: \fi
\begin{itemize}
\item overflow \ifanswers se $x$ esterno a \else if $x$ does not belong to \fi $[-M_1,M_1]$
\item underflow \ifanswers se $x$ interno a \else if $x$ belongs to \fi $(-m_1,m_1)$ 
\ifanswers (approssimato con lo zero) \else rounded by the zero \fi
\end{itemize}}
%Nel formato float, se tutti i bit dell'esponente sono uno, casi speciali
%\vskip 2mm
%Es. $0|11111111|00000000000000000000000\to + \infty$
%Es. $0|11111111| 00000000001000001000000 \to NaN$}
\end{document}
