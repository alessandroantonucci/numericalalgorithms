\documentclass[professionalfonts]{beamer}
\newif\ifita
\itatrue % comment out to hide answers
%\itafalse
\usepackage[familydefault,light]{Chivo} 
\usepackage[T1]{fontenc}
\usenavigationsymbolstemplate{}
\usepackage[]{hyperref}
\usepackage{tikz,pgf,pgfarrows,pgfnodes,pgfbaseimage}
\graphicspath{{./Pics/}}
\usetikzlibrary{shapes}
\usepackage{setspace}
\newcommand{\evi}[1]{{\colorbox{yellow!50}{{#1}}}}
\newcommand{\exe}[1]{{\color{black!50}{{#1}}}}
\newcommand{\kw}[1]{{\colorbox{black!30}{\color{white}{#1}}}}
\tikzstyle{nd}=[circle,draw=black,thick,minimum size=.8cm,inner sep=1pt]
\setbeamercovered{transparent}
\usetheme{Singapore}
\tikzstyle{nodo}=[ellipse,draw=black!60,fill=black!10,line width=.7pt,minimum width=.7cm,minimum height=.4cm]
\usecolortheme[named=gray]{structure}
\setbeamercolor{block title}{bg=black!20,fg=black}
\setbeamercolor{block body}{bg=black!10,fg=black}

%%%%%%%%%%%%%%%%%%%%%
\ifita
\title{Algoritmi Numerici (Parte III)}
\subtitle{[Lezione 2] Metodo di Eliminazione di Gauss}
\else
\title{Numerics (Part III)}
\subtitle{[Lecture 2] Gaussian Elimination Algorithm}
\fi
\date{}
\author{Alessandro Antonucci\\{\tt alessandro.antonucci@supsi.ch}}
\begin{document}
\maketitle
\setstretch{1.4}
%%%%%%%%%%%%%%%%%%%%%%%%%%%%
\frame{\frametitle{\ifita Algoritmo di Sostituzione \else Solving by Substitution \fi}
\ifita
Sistema triangolare si risolve per \evi{sostituzione} (dal basso verso l'alto) 
\else
Triangular system solved by (bottom-up) \evi{substitution}
\fi
$$\hat{A}=\left[ \begin{array}{ccc}  1 & -1 & 2\\ 0  & -1 & -1 \\ 0 & 0 & 1  \end{array} \right]
\quad
\vec{x} = 
\left[ \begin{array}{c}  x_1 \\  x_2 \\  x_3 \end{array} \right]
\quad
\vec{b} = 
\left[ \begin{array}{c}  2 \\  0 \\  3 \end{array} \right]$$
\begin{columns}
\begin{column}[]{0.3\textwidth} 
$$\left\{ \begin{array}{c}  x_1 - x_2 + 2 x_3 = 2 \\ \phantom{2x_1}-x_2-x_3=0  \\ \phantom{x_1+x_2+}x_3 = 3  \end{array} \right.$$
\end{column}
\begin{column}[]{0.7\textwidth} 
$$\left\{ \begin{array}{ll}  x_1 =& 2 + x_2 - 2 x_3 = 2-3-6=-7  \\ x_2 = &-x_3 = -3  \\ x_3 =& 3  \end{array} \right.$$
\end{column}
\end{columns}
\vskip 2mm
\centering
\ifita complessit\`a $O(N^2)$ \else $O(N^2)$ complexity \fi
}
%%%%%%%%%%%%%%%%%%%%%%%
\frame{\frametitle{\ifita Problema Generale \else General Problem \fi}
\ifita Dato un sistema qualunque (= non triangolare)
\else General (non-triangular) linear system
\fi 
$$\hat{A}=\left[ \begin{array}{ccc}  1 & -1 & 2\\ 2  & -1 & -1 \\ 1 & 1 & 1  \end{array} \right]
\quad
\vec{x} = 
\left[ \begin{array}{c}  x_1 \\  x_2 \\  x_3 \end{array} \right]
\quad
\vec{b} = 
\left[ \begin{array}{c}  2 \\  0 \\  3 \end{array} \right]$$

\begin{itemize}
\ifita
\item IDEA: trasformare sistema in triangolare \evi{equivalente}
\item Equivalente = stesse soluzioni
\item Trasformazioni equivalenti = cambiano aspetto sistema\\ non le soluzioni
\else
\item IDEA: Make the system a \evi{equivalent} linear system
\item Equivalent = same solutions
\item Equivalent transformations = system changes aspect\\ but the solution is the same
\fi
\end{itemize}}



%%%%%%%%%%%%%%%%%%%%%%%%%%%%%%%%%%
\frame{\frametitle{\ifita Due Trasformazioni Equivalenti \else Two Equivalent Transformations \fi}\small
\begin{columns}
\begin{column}[T]{0.5\textwidth} 
$\left\{ \begin{array}{c}  x_1 - x_2 + 2 x_3 = 2 \\ 2x_1-x_2-x_3=0  \\ x_1+x_2+x_3 = 3  \end{array} \right.$
\end{column} 
\begin{column}[T]{0.5\textwidth} 
``Box'' \quad$\begin{array}{|ccc|c|}  \hline 1 & -1 & 2 &2\\ 2  & -1 & -1 &0 \\ 1 & 1 & 1 &3\\ \hline \end{array}$
\end{column}
\end{columns}
\begin{itemize}
\item Swap row 1 $\iff$ row 2
\end{itemize}
$\left\{ \begin{array}{c} 2x_1-x_2-x_3=0  \\   x_1 - x_2 + 2 x_3 = 2 \\   x_1+x_2+x_3 = 3  \end{array} \right.$
$\begin{array}{|ccc|c|}  \hline 2  & -1 & -1 &0 \\ 1 & -1 & 2 &2\\ 1 & 1 & 1 &3\\ \hline \end{array}$
\begin{itemize}
    \item $3.0$ $\times$ row 2
\end{itemize}
$\left\{ \begin{array}{c}  x_1 - x_2 + 2 x_3 = 2 \\ 6x_1-3x_2-3x_3=0  \\ x_1+x_2+x_3 = 3  \end{array} \right.$
$\begin{array}{|ccc|c|}  \hline 1 & -1 & 2 &2\\ 6  & -3 & -3 &0 \\ 1 & 1 & 1 &3\\ \hline \end{array}$
}

\frame{\frametitle{\ifita Trasformazioni Equivalenti\else Equivalent Transformations \fi}
\begin{columns}
\begin{column}[T]{0.5\textwidth} 
IF $a = b$
\vskip 2mm
THEN $a + 2 = b + 2$
\vskip 4mm
IF $a = b$ e $c = d$ \vskip 2mm
THEN $a + c = b + d$
\end{column}
\begin{column}[T]{0.5\textwidth} 
\begin{block}{\ifita REGOLE \else RULES \fi}
\begin{itemize}
\ifita
\item Riga per costante
\item Riga meno (pi\`u) altra riga
\item Swap righe
\else
\item Row by constant
\item Row minus (plus) row
\item Swapping Rows
\fi
\end{itemize}
\end{block}
\end{column}
\end{columns}
\vskip 4mm
\begin{itemize}
\ifita
\item Posso sommare ad un'equazione un'altra
equazione senza che la soluzione del sistema cambi
\item Gauss = trasformazioni equivalenti
per rendere triangolare un sistema qualunque
\else
\item We can sum an equation to another one\\without changing the solution
\item Gauss = equivalent transformations\\
to make triangular a generic system
\fi
\end{itemize}}


\frame{\frametitle{Gauss (2$\times$2)}
\begin{columns}
\begin{column}[T]{0.7\textwidth} 
\begin{itemize}
\ifita
\item Rendere il sistema triangolare? \emph{Annichilire} (rendere nullo) l'elemento dalla seconda riga prima colonna
\item Sottraggo alla seconda riga la prima moltiplicata per $k$
\item Con $k=\frac{2}{5}$ l'elemento in riga2-col1 si annichila
\else
\item Make the system triangular? \emph{annihilate} (make zero) the element in the second row and first column
\item Second row minus first row multiplied by $k$
\item With $k=\frac{2}{5}$ the element in row2-col1 becomes zero
\fi
\end{itemize}
\end{column}
\begin{column}[T]{0.3\textwidth} 
$$\begin{array}{|cc|c|}  \hline 5 & 2 & -3\\ 2  & 1 & 6 \\ \hline \end{array}$$
$$\begin{array}{|cc|c|}  \hline 5 & 2 & -3\\ 0  & 1-\frac{4}{5} & 6+\frac{6}{5} \\ \hline \end{array}$$
$$\begin{array}{|cc|c|}  \hline 5 & 2 & -3\\ 0  & \frac{1}{5} & \frac{36}{5} \\ \hline \end{array}$$
\end{column}
\end{columns}
}


\frame{\frametitle{Gauss (3$\times$3)}
\begin{columns}
\begin{column}[T]{0.5\textwidth}
\begin{itemize}
\item
\ifita
Annichilisco i (due) elementi non diagonali della prima colonna
\else
Annihilating the two non-diagonal elements of the first column
\fi
\end{itemize}
\end{column}
\begin{column}[T]{0.5\textwidth} 
$\begin{array}{|ccc|c|}  \hline 2  & -1 & -1 &0 \\ 1 & -1 & 2 &2\\ 3 & 1 & 1 &3\\ \hline \end{array}$
\end{column}
\end{columns}
\vskip 3mm
\begin{columns}
\begin{column}[T]{0.5\textwidth} 
\begin{itemize}
\item
\ifita Seconda riga meno prima moltiplicata per \else Second row minus first multiplied by \fi $k=\frac{1}{2}$
\end{itemize}
\end{column}
\begin{column}[T]{0.5\textwidth} 
$\begin{array}{|ccc|c|}  \hline 2  & -1 & -1 &0 \\ 0 & -\frac{1}{2} & \frac{5}{2} &2\\ 3 & 1 & 1 &3\\ \hline \end{array}$
\end{column}
\end{columns}
\vskip 3mm
\begin{columns}
\begin{column}[T]{0.5\textwidth} 
\begin{itemize}
    \item
\ifita Terza riga meno prima moltiplicata per 
\else Third row minust first multiplied by \fi
$k=\frac{3}{2}$
\end{itemize}
\end{column}
\begin{column}[T]{0.5\textwidth} 
$\begin{array}{|ccc|c|}  \hline 2  & -1 & -1 &0 \\ 0 & -\frac{1}{2} & \frac{5}{2} &2\\ 0 & \frac{5}{2} & \frac{5}{2} &3\\ \hline \end{array}$
\end{column}
\end{columns}
\centering
\begin{itemize}
{\small
\item \ifita
Sistema non ancora triangolare, ma ultime due righe\\sono sistema 2$\times$2, 
agisco ricorsivamente!
\else
Not triangular, but last rows are $2\times 2$ system, go recursive!
\fi}
\end{itemize}

}
\frame{\frametitle{Gauss in Python}
\lstinputlisting[
language=Python,
firstline=0,
lastline=10
]{./Pics/gauss.py}
%\lstinputlisting[language=Java]{a.java}
\vskip 2mm
\begin{center}
\ifita Tre cicli for nidificati, complessit\`a $O(n^3)$
\else Three nested for loops, complexity $O(n^3)$\fi
\end{center}
}

\frame{\frametitle{\ifita Scelta del Pivot \else Picking the Pivot \fi}
\setstretch{1.4}
\begin{itemize}
\ifita
\item Se ${\tt pivot=0}$, ${\tt coeff}$ non pu\`o essere calcolato
\item Se ${\tt pivot}$ \`e piccolo, ${\tt coeff}$ \`e grande
\item L'equazione del {\tt pivot} viene moltiplicata per ${\tt coeff}$,\\ ${\tt coeff}$ grande introduce imprecisione nella rappresentazione
\item Swap di riga e colonna permettono di cambiare il pivot
\item Swap pivot $A_{kk}$ con elemento pi\`u grande in valore assoluto $A_{i^*j^*}$ 
\else
\item If ${\tt pivot=0}$, ${\tt coeff}$ cannot be computed
\item If ${\tt pivot}$ is small, ${\tt coeff}$ is big
\item {\tt pivot} equation multiplied by ${\tt coeff}$,\\ big ${\tt coeff}$ add imprecision in the representation
\item Swapping rows and/or cols allow to change the pivot
\item Swapping pivot $A_{kk}$ with biggest value (in abs value) $A_{i^*j^*}$ 
\fi
$$(i^*,j^*) = \arg \max_{\substack{i=k,\ldots,n \\ j=k,\dots,n}} |A_{ij}|$$
\end{itemize}}
\end{document}
