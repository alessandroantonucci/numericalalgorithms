\documentclass[professionalfonts]{beamer}
\newif\ifita
%\itatrue
\itafalse
\usepackage[familydefault,light]{Chivo} 
\usepackage[T1]{fontenc}
\usenavigationsymbolstemplate{}
\usepackage[]{hyperref}
\usepackage{tikz,pgf,pgfarrows,pgfnodes,pgfbaseimage}
\graphicspath{{./Pics/}}
\usetikzlibrary{shapes}
\usepackage{setspace}
\newcommand{\evi}[1]{{\colorbox{yellow!50}{{#1}}}}
\newcommand{\exe}[1]{{\color{black!50}{{#1}}}}
\newcommand{\kw}[1]{{\colorbox{black!30}{\color{white}{#1}}}}
\tikzstyle{nd}=[circle,draw=black,thick,minimum size=.8cm,inner sep=1pt]
\setbeamercovered{transparent}
\usetheme{Singapore}
\tikzstyle{nodo}=[ellipse,draw=black!60,fill=black!10,line width=.7pt,minimum width=.7cm,minimum height=.4cm]
\usecolortheme[named=gray]{structure}
\setbeamercolor{block title}{bg=black!20,fg=black}
\setbeamercolor{block body}{bg=black!10,fg=black}

\ifita
\title{Algoritmi Numerici (Parte II)}
\subtitle{[Lezione 6] Algoritmo Passo-Passo}
\else
\title{Numerics (Part II)}
\subtitle{[Lecture 6] Step-by-Step Algorithm}
\fi
\date{}
\author{Alessandro Antonucci\\{\tt alessandro.antonucci@supsi.ch}}
\begin{document}
    \maketitle
    \setstretch{1.4}
%\date{\tiny\url{https://colab.research.google.com/drive/1noTW7iMitznYtHWrOgBTVkvEZzev5oEi}}
%%%%%%%%%%%%%%%%%%%%%%%%%%%%
\frame{\frametitle{Step-by-Step Algorithm}
\begin{itemize}
\ifita
\item A bisez e RF serve intervallo $[a,b]$ t.c. $f(a)\cdot f(b) < 0$
\item Per trovarlo, divido in $k$ intervalli di uguale ampiezza la regione $[x_{\mathrm{min}},x_{\mathrm{max}}]$ su cui \`e definita $f$
\item Ampiezza intervalli \`e $\Delta = \frac{x_{\mathrm{max}}-x_{\mathrm{min}}}{k}$
\item Scansiono intervalli da sx a dx, stop quando cambia segno 
\else
\item Bisection and RF need an interval $[a,b]$ s.t. $f(a)\cdot f(b) < 0$
\item Finding it in a (large) interval $[x_{\mathrm{min}},x_{\mathrm{max}}]$\\ where $f$ is defined
\item Split interval $[x_{\mathrm{min}},x_{\mathrm{max}}]$ in $k$ equally-sized sub-intervals
\item (Sub)Intervals width is $\Delta = \frac{x_{\mathrm{max}}-x_{\mathrm{min}}}{k}$
\item Scanning (sub)intervals from left to right,\\ stop when sign change detected
\fi
\end{itemize}}
\frame{\frametitle{\ifita Metodo passo-passo \small(continua)\else Step-by-Step Algorithm (Con't)\fi}
\begin{itemize}
\item $\Delta = \frac{x_{\mathrm{max}}-x_{\mathrm{min}}}{k}$
\item How to choose the ``right'' $\Delta$?
\begin{itemize}
\item Small $\Delta$? Too large $k$ (slow)
\item Big $\Delta$? Might skip sign-change regions
\end{itemize}
%\ifita
%\item In pratica si parte da $\Delta$ grandi e poi,\\
%se non si trova un cambio di segno,\\si riduce (es. 
%$\Delta = \Delta / 10$)
%\else
\item In practice: 
\begin{itemize}
\item start with a large $\Delta$
\item run step-by-step algorithm from left to right
\item if no sign-change detected at end ($x_\mathrm{max}$) 
\item back to the left ($x_\mathrm{min}$) and reduce $\Delta$ \\(ex. $\Delta = \Delta / 10$)
\end{itemize}
%\fi
\end{itemize}}
\end{document}

\frame{\frametitle{Metodo babilonese per calcolo $\sqrt{k}$}
\begin{itemize}
\item Ricorsione di ordine uno:
\begin{center}
$x_{j+1} = \frac{1}{2} \left( x_j + \frac{k}{x_j} \right)$
\end{center}
\item Dato un valore iniziale $x_0>0$ converge a $\sqrt{k}$
\item Es. $k=2$, $x_0=2$, $x_1=1.5$, $x_2=1.41\overline{6}$, $x_3=1.4142356\ldots \simeq \sqrt{2}$ 
\item Analogie con metodo della tangente
\item Pu\`o essere interpretato come un algoritmo\\per trovare gli zeri di una funzione?
\end{itemize}}

\frame{\frametitle{Perch\'e funziona?}
\begin{itemize}
\item Riscrivo la ricorsione senza gli indici
\begin{center}
$x = \frac{1}{2} \left( x + \frac{k}{x} \right) \Rightarrow x^2-k=0$\\
{\color{gray!50}{\small $-x+\frac{1}{2}\left(x+\frac{k}{x} \right) = \frac{x^2 - 2x^2 + k}{2x} = \frac{-x^2 + k}{2x}$}}
\end{center}
\item $f(x):= x^2-k$ ha come zero $x^*=\sqrt{k}$
\item Il metodo babilonese scrive $f(x)=0$ come $x=g(x)$
\item L'equazione letta come una ricorsione $x_{j+1} = g(x_j)$
\item A convergenza $x^* = g(x^*)$
\end{itemize}
}

\frame{\frametitle{Iterazione di punto fisso}
\begin{itemize}
\item Rovesciando considerazioni su metodo babilonese,\\abbiamo metodo generale per trovare zeri $f$
\begin{itemize}
\item Data un'equazione $f(x)=0$ riscriviamola come $x=g(x)$
\item Ricorsione di ordine uno $x_{j+1} = g(x_j)$
\item Convergenza implica $x^*=g(x^*)$ e quindi $f(x^*)=0$
\end{itemize}
\item L'iterazione di punto fisso non sempre converge
\item $g(x):=x+f(x)$ \`e il modo pi\`u immediato di ottenere $g$,\\ ma non l'unico n\'e il migliore
\end{itemize}}


\frame{\frametitle{Interpretazione geometrica iterazione punto fisso}
\begin{columns}
\begin{column}{0.7\textwidth}
\begin{tikzpicture}[scale=2.5,domain=1.2:2]
\draw[color=red] plot[id=pa,domain=.5:2] function{(x+2.0/x)/2.0} node[right] {\tiny $g(x)=\frac{1}{2}(x+\frac{2}{x})$};
\draw[color=blue] plot[id=pb,domain=0:2] function{x} node[right] {\tiny $y=x$};
\filldraw [black] (2,0) circle (.5pt) node[below] {\tiny $x_0$};
\filldraw [black] (1.5,0) circle (.5pt) node[below left] {\tiny $x_1$};
\filldraw [black] (1.4166,0) circle (.5pt) node[below right] {\tiny $x_2$};
\filldraw [green] (2,1.5) circle (.5pt);
\filldraw [green] (1.5,1.5) circle (.5pt);
\filldraw [green] (1.5,1.4166) circle (.5pt);
\filldraw [green] (1.4166,1.4166) circle (.5pt);
\draw [->] (0,0) -- (2.1,0);
\draw [->] (0,0) -- (0,2.1) node[above] {\tiny $y$};
\draw [black!20] (2,0) -- (2.0,1.5);
\draw [black!20] (2.0,1.5) -- (1.5,1.5);
\draw [black!20,dotted] (1.5,1.5) -- (1.5,0);
\draw [black!20,dotted] (1.4166,1.4166) -- (1.4166,0);
\draw [black!20] (1.5,1.5) --(1.5,1.4166);
\end{tikzpicture}
\end{column}
\begin{column}{0.3\textwidth}
\centering
Il grafico di $g(x)$ \\
incontra la bisettrice in $x^*$\\
infatti\\
$x^* = g(x^*)$\\
se e solo se\\
$f(x^*)=0$
\end{column}
\end{columns}}

\frame{\frametitle{Esercizi}
\begin{enumerate}
\item $f(x)=\frac{2+x}{2}\sqrt{2x}-12$, $x_0=5$, {\color{gray}{$g(x)=\frac{24}{\sqrt{2x}}-2$}}
\item $f$ e $x_0$ come sopra ma 
{\color{gray}{$g(x)=\frac{1}{2} \left( \frac{24}{2+x} \right)^2$}}
\item Risolvere l'equazione $x+\ln x = 0$
\end{enumerate}}
\end{document}
