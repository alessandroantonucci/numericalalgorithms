\documentclass[a4paper,12pt]{amsart}
\usepackage[a4paper]{geometry}
\geometry{ a4paper, total={170mm,257mm}, left=20mm, top=20mm, }
\usepackage[familydefault,light]{Chivo}
\usepackage{setspace}
\newif\ifita
\itatrue % comment out to hide answers
\itafalse
\begin{document}
\onehalfspacing
\renewcommand{\baselinestretch}{1.5} 
\clearpage\thispagestyle{empty}
\begin{center}
{\Large \bf ::: \ifita FORMULARIO \else FORMULAE \fi :::}
\end{center}
\vskip 3mm
\paragraph{\bf ::: \ifita Premessa \else Introduction \fi :::}
\begin{itemize} 
\item Input: $\{x_0,x_1,\ldots,x_n\}$ e $\{y_0,y_1,\ldots,y_n\}$ (\ifita coordinate dei punti \else points coordinates\fi).
\item $n+1$ \ifita punti che definiscono \else points defining \fi $n$ \ifita intervalli \else intervals \fi.
\end{itemize}
\vskip 3mm
\paragraph{\bf ::: \ifita Interpolazione polinomiale \else Polynomial interpolation \fi:::}
\begin{itemize} 

\item \ifita Polinomio interpolante di grado \else Interpolating polynomial of degree \fi $n$:
$$p(x):=\sum_{i=0}^n c_i \left[ \prod_{j=0}^{i-1} (x-x_j) \right]$$
\ifita Es. se $n=2$ (3 punti), \else Ex. if $n=2$ (3 points), \fi $p(x)=c_0+c_1(x-x_0)+c_2(x-x_0)(x-x_1)$.
\ifita
\item I parametri $\{c_0,c_1,\ldots,c_n\}$ si cacolano (in tempo quadratico e spazio lineare) con lo schema di calcolo a piramide.
\else
\item Parameters $\{c_0,c_1,\ldots,c_n\}$ are obtained (in quadratic time and linear space) by the pyramid-shape computations.
\fi
\end{itemize}

\subsection*{\bf ::: \ifita Spline quadratica \else Quadratic spline \fi :::}
\begin{itemize}
\ifita
\item Per $x\in[x_k,x_{k+1}]$, ho la parabola:
\else
\item For $x\in[x_k,x_{k+1}]$, the parabolic curve rewrites as:
\fi
$$f_k(x):= y_k+z_k(x-x_k)+\frac{z_{k+1}-z_k}{2(x_{k+1}-x_k)} (x-x_k)^2$$
\ifita
per ogni $k=0,...,n-1$ (complessivamente $n$ parabole).
\else
for each $k=0,...,n-1$ ($n$ parabolic curves overall).
\fi
\ifita
\item I parametri $z_0,\ldots,z_n$ (intepretazione $z_k:=f'(x_k)$) si ricavano con ricorsione di ordine uno (complessit\`a lineare).
\else
\item Parameters $z_0,\ldots,z_n$ (intepretation $z_k:=f'(x_k)$) are obtained by an order-1 recursion  (linear complexity).
\fi
\ifita
\item Il valore iniziale di $z_0$ \`e assegnato, per gli altri:
\else
\item Starting value $z_0$ is given, for the others:
\fi
$$z_{k+1} = -z_k + 2 \frac{y_{k+1}-y_k}{x_{k+1}-x_k}$$
\end{itemize}

\subsection*{\bf ::: \ifita Spline cubica (naturale) \else Cubic spline (natural) \fi :::} 
\begin{itemize}
\ifita
\item Per $x\in[x_k,x_{k+1}]$, ho la cubica:
\else
\item With$x\in[x_k,x_{k+1}]$, the cubic interpolating function is:
\fi
$$f_k(x)=\frac{z_{k+1}(x-x_k)^3+z_k(x_{k+1}-x)^3}{6h_k}+\left(\frac{y_{k+1}}{h_k}-\frac{h_k}{6}z_{k+1}\right)(x-x_k)+\left(\frac{y_{k}}{h_k}-\frac{h_k}{6}z_{k}\right)(x_{k+1}-x),$$
\ifita
per ogni $k=0,...,n-1$, dove $h_k:=(x_{k+1}-x_k)$ reppresenta l'ampiezza dell'intervallo su cui \`e definita la cubica.
\else
for each $k=0,...,n-1$, where $h_k:=(x_{k+1}-x_k)$ represents the width on the interval where the cubic function is defined.
\fi
\ifita
\item I parametri $z_0,\ldots,z_n$ (intepretazione $z_k:=f''(x_k)$) si ricavano mediante risoluzione di un sistema lineare di $n-1$ equazioni nelle $n-1$ incognite $z_1,\ldots,z_{n-1}$, mentre $z_0=z_n=0$ per la condizione di naturalit\`a. Le $n-1$ equazioni del sistema sono:
\else
\item Parameters $z_0,\ldots,z_n$ (intepretation $z_k:=f''(x_k)$) are obtained from a linear system of $n-1$ equations with respect to the $n-1$ unknowns $z_1,\ldots,z_{n-1}$, while $z_0=z_n=0$ because of the natural sp-line condition. The $n-1$ equations of the linear system are:
\fi
$$h_{k-1}z_{k-1}+2(h_{k-1}+h_k)z_k+h_kz_{k+1}=6\left(\frac{y_{k+1}-y_k}{h_k}-\frac{y_k-y_{k-1}}{h_{k-1}}\right),$$
\ifita con $k=1,\ldots,n-1$. Se $h_k=1$ per ogni $k$ (ovvero le $x$ dei punti sono equidistanti ed a distanza uno), le equazioni si riscrivono come:
\else
where $k=1,\ldots,n-1$. If $h_k=1$ for each $k$ (that means the $x$ of the points are at the same distance on the x-axis and at distance one), equations rewrite as:
\fi
$$z_{k-1}+4 z_k+z_{k+1}=6 (y_{k+1}-2y_k+y_{k-1})$$
\end{itemize}
\newpage
\subsection*{\bf ::: \ifita Interpolazione Trigonometrica \else Trigonometric Interpolation \fi :::} 
\begin{itemize}
\item \ifita Premessa: in questa formula, ho $n$ punti (e non $n+1$ come nelle altre) di coordinate $\{(x_i,y_i)\}_{i=0}^{n-1}$ ed assumo $n$ dispari.
\else
Premise: in this formula we have $n$ points (and not $n+1$ as in the other formulae) of coordinates $\{(x_i,y_i)\}_{i=0}^{n-1}$ and we assum $n$ to be odd.
\fi
\item \ifita La funzione interpolante \`e \else The interpolating function is\fi:
$$f(x):= C + \sum_{j=1}^m \left[ a_j \cos (jx) + b_j \sin (jx) \right],$$
\ifita
e dipende da $n$ parametri (quindi $m=\frac{n-1}{2}$).
\else
and depends on $n$ parameters (thus, $m=\frac{n-1}{2}$).
\fi
\ifita
\item Se i punti di appoggio dividono l'intervallo $[0,2\pi]$ in $n$ parti uguali,\\allora valgono le seguenti formule:
\else
\item If the points are splitting the interval $[0,2\pi]$ in $n$ parts of equal size,\\the following formulae hold:
\fi
$$a_j = \frac{2}{n} \sum_{i=0}^{n-1} \left[ y_i \cos(j x_i) \right], \quad \forall j=0,1,\ldots,m$$
$$b_j = \frac{2}{n} \sum_{i=0}^{n-1} \left[ y_i \sin(j x_i) \right], \quad \forall j=1,\ldots,m,$$
\ifita
e $C=\frac{a_0}{2}$, con $a_0$ che si ricava dalla formula dei coseni (che da' semplicemente $a_0=\frac{2}{n}\sum_{i=0}^{n-1} y_i$, ovvero $C$ \`e il valore medio delle $y$).
\else
and $C=\frac{a_0}{2}$, where $a_0$ is obtained from the cosine formula\\ (this giving $a_0=\frac{2}{n}\sum_{i=0}^{n-1} y_i$, that means $C$ is the average of the $y$) coordinates.
\fi
\end{itemize}

\subsection*{\bf ::: \ifita Regressione lineare \else Linear Regression \fi :::} 
\begin{itemize}
\item \ifita Dati $n$ punti \else Given $n$ points \fi $\{(x_i,y_i)\}_{i=1}^{n}$ \ifita trovo retta rappresentativa di equazione \else finding representative line of equation \fi $y=mx+q$
\item $m=\frac{\sum_i x_i y_i - n \overline{x} \, \overline{y}}{\sum_i x_i^2 - n \overline{x}^2}$ \ifita e \else and \fi $q = \overline{y} - m \overline{x}$ \ifita con \else where \fi $\overline{x} = \frac{\sum_i x_i}{n}$ \ifita e \else and \fi $\overline{y}=\frac{\sum_i y_i}{n}$
\end{itemize}


\subsection*{\bf ::: \ifita Integrazione Numerica \else Numerical Integration \fi :::} 
\begin{itemize}
\item \ifita Approssimo l'integrale \else Approximating integral \fi $\int_{a}^b f(x) \mathrm{d}x$ \ifita dividendo l'intervallo \else by dividing interval \fi $[a,b]$ in $n$ \ifita parti mediante gli \else parts by \fi $n+1$ \ifita punti \else points \fi $x_0,x_1,\ldots,x_n$ \ifita con \else where \fi $x_0=a$ e $x_n=b$.
\item \ifita Somma aree trapezi \else Summing trapezoids areas \fi $A_{tr}=\sum_{i=0}^{n-1} (x_{i+1}-x_i) \frac{f(x_i)+f(x_{i+1})}{2}=\frac{b-a}{n} \cdot \sum_{i=0}^{n-1} \frac{f(x_i)+f(x_{i+1})}{2}$
\end{itemize}


\end{document}
