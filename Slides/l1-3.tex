\documentclass[professionalfonts]{beamer}
\newif\ifanswers
%\answerstrue % comment out to hide answers
\answersfalse
\usepackage[familydefault,light]{Chivo} 
\usepackage[T1]{fontenc}
\usenavigationsymbolstemplate{}
\usepackage[]{hyperref}
\usepackage{tikz,pgf,pgfarrows,pgfnodes,pgfbaseimage}
\graphicspath{{./Pics/}}
\usetikzlibrary{shapes}
\usepackage{setspace}
\newcommand{\evi}[1]{{\colorbox{yellow!50}{{#1}}}}
\newcommand{\exe}[1]{{\color{black!50}{{#1}}}}
\newcommand{\kw}[1]{{\colorbox{black!30}{\color{white}{#1}}}}
\tikzstyle{nd}=[circle,draw=black,thick,minimum size=.8cm,inner sep=1pt]
\setbeamercovered{transparent}
\usetheme{Singapore}
\tikzstyle{nodo}=[ellipse,draw=black!60,fill=black!10,line width=.7pt,minimum width=.7cm,minimum height=.4cm]
\usecolortheme[named=gray]{structure}
\setbeamercolor{block title}{bg=black!20,fg=black}
\setbeamercolor{block body}{bg=black!10,fg=black}

\ifanswers
\title{Algoritmi Numerici (Parte I)}
\subtitle{[Lezione 3] Numeri Razionali  $\mathbb{Q}$, Algoritmo di Horner per Numeri Frazionari e suo Inverso}
\else
\title{Numerics (Part I)}
\subtitle{[Lecture 3] Rational Numbers $\mathbb{Q}$,\\Horner's Algorithm for Fractional Numbers and Inverse}
\fi
\date{}
\author{Alessandro Antonucci\\{\tt alessandro.antonucci@supsi.ch}}
%%%%%%%%%%%%%%%%%%%%%%%%%%%%
%\setstretch{1.4}
%\documentclass[professionalfonts]{beamer}
%\documentclass[professionalfonts,handout]{beamer}
%\usepackage[familydefault,light]{Chivo} 
\usepackage[T1]{fontenc}
\usenavigationsymbolstemplate{}
\usepackage[]{hyperref}
\usepackage{tikz,pgf,pgfarrows,pgfnodes,pgfbaseimage}
\graphicspath{{./Pics/}}
\usetikzlibrary{shapes}
\usepackage{setspace}
\newcommand{\evi}[1]{{\colorbox{yellow!50}{{#1}}}}
\newcommand{\exe}[1]{{\color{black!50}{{#1}}}}
\newcommand{\kw}[1]{{\colorbox{black!30}{\color{white}{#1}}}}
\tikzstyle{nd}=[circle,draw=black,thick,minimum size=.8cm,inner sep=1pt]
\setbeamercovered{transparent}
\usetheme{Singapore}
\tikzstyle{nodo}=[ellipse,draw=black!60,fill=black!10,line width=.7pt,minimum width=.7cm,minimum height=.4cm]
\usecolortheme[named=gray]{structure}
\setbeamercolor{block title}{bg=black!20,fg=black}
\setbeamercolor{block body}{bg=black!10,fg=black}

%\title{Algoritmi Numerici (Parte I)}
%\subtitle{[Lezione 3] Numeri razionali}
%\author{Alessandro Antonucci\\{\tt alessandro.antonucci@supsi.ch}}
%\date{\tiny\url{https://colab.research.google.com/drive/1Vz6HUgSJLjvMZCDgFlO8nrMYXECa310t}}
%%%%%%%%%%%%%%%%%%%%%%%%%%%%
\begin{document}
\maketitle
\frame{\frametitle{\ifanswers I Numeri Razionali e la Divisione\else Rational Numbers and Division \fi}
\setstretch{1.4}
\begin{itemize}
\ifanswers
\item Moltiplicazione \`e addizione iterata (col segno per interi)
\else
\item Multiplication as an iterated sum (signed for integers)
\fi
\ifanswers
\item \evi{Divisione} operazione inversa alla moltiplicazione
\else
\item \evi{Division}? Inverse multiplication!
\fi
\ifanswers
\item L'insieme degli interi non \`e chiuso rispetto alla divisione
$a,b\in\mathbb{Z}$ $\Rightarrow$ $\mathrm{DIVISIONE}(a,b) \in \mathbb{Z}$ solo se $|a|$ multiplo di $|b|$
\else
\item Integers not closed wrt division\\
$a,b\in\mathbb{Z}$ $\Rightarrow$ $\mathrm{DIVISION}(a,b) \in \mathbb{Z}$ if and only if $|a|=k|b|$
\fi
\ifanswers
\item Soluzione? Allargare l'insieme definendo $a/b$
\else
\item Key? Increase the set and define new number $a/b$
\fi
\ifanswers
\item L'insieme allargato $\mathbb{Q}$ si chiama dei numeri \evi{razionali}
\else
\item Larger set is $\mathbb{Q}$: \evi{rational} numbers
\fi
\ifanswers
\item I razionali si esprimono com rapporti di numeri interi (primi fra loro, $a/b$ e $(ka)/(kb)$ sono lo stesso numero)
\else
\item Rational numbers are ratios of integer number\\ (note $a/b$ and $(ka)/(kb)$ are the same number)

\fi
\end{itemize}}



\frame{\frametitle{\ifanswers Quanti sono i numeri razionali?\else How many rational numbers?\fi} 
%\begin{columns}
%\begin{column}[T]{0.8\textwidth}
\begin{itemize}
\item \ifanswers Infiniti, ma in corrispondenza uno a uno con i naturali\else Infinite, but 1-to-1 correspondence with $\mathbb{N}$\fi
\item \ifanswers Ogni intero \`e razionale \else Every integer is rational \fi ($\mathbb{Z}\subseteq \mathbb{R}$) 
\ifanswers
\exe{Es. 135 = 135/1}
\else
\exe{Ex. 135 = 135/1}
\fi
\ifanswers
\item Elementi $\mathbb{Q}$ su matrice con infinite righe e colonne
\else
\item Elements of $\mathbb{Q}$ on a matrix with infinite rows and cols
\fi
\vskip 2mm
\ifanswers
$a/b \in \mathbb{Q}$ con $b>0$ su elemento di riga $a$ e colonna $b$
\else
$a/b \in \mathbb{Q}$ with $b>0$ on element with row $a$ and col $b$
\fi
\vskip 2mm
\begin{tabular}{ccccccccc}
$\ldots$ & $\ldots$ & $\ldots$ & $\ldots$ & $\ldots$ & $\ldots$ & $\ldots$ & $\ldots$ \\
$\ldots$ & -3/3&  -2/3&  -1/3&   0/3&   1/3&   2/3&   3/3 & $\ldots$\\
$\ldots$ & -3/2&  -2/2&  -1/2&   0/2&   1/2&   2/2&   3/2 & $\ldots$\\
$\ldots$ & -3/1&  -2/1&  -1/1&   0/1&   1/1&   2/1&   3/1 & $\ldots$
\end{tabular}
\end{itemize}
\begin{center}
\ifanswers
Algoritmo di ricopertura (a spirale) 
\vskip 1mm
produce corrispondenza con i naturali
\else
Spiral coverage algorithm
defines\\correspondence with natural numbers
\fi
\end{center}
}

\frame{\frametitle{\ifanswers Rappresentazione razionali in base 10\else Base 10 representation of rational numbers\fi}
\begin{center}
\ifanswers
Se $a/b$ (non necessariamente ai minimi termini) tale che $b=10^k$
\else
If $a/b$ (not necessarily reduce to lowest terms) such that $b=10^k$
\fi
\vskip 1mm
\ifanswers
il numero ha una rappresentazione decimale (finita)
\else
the number has a decimal (and finite) representation
\fi
\vskip 2mm
Es. $134.75 = \frac{13475}{100}$
\end{center}
\ifanswers
Somma potenze positive (sx del punto) e negative (dx)
\else
Sum of positive (left wrt dot) and negative (right wrt dot) powers
\fi
$134.75 = 134 + .75$
\\
$134= 1 \cdot 10^2 + 3 \cdot 10^1 + 4 \cdot 10^0$ \\$0.75 = 7 \cdot 10^{-1} + 5 \cdot 10^{-2}$
\ifanswers
\begin{block}{Due Osservazioni}
\begin{enumerate}
\item La stessa cosa si pu\`o fare con basi diverse da 10
\item I numeri si possono leggere con l'algoritmo di Horner\\
(adattato al caso di potenze negative)
\end{enumerate}
\end{block}
\else
\begin{block}{Two remarks}
    \begin{enumerate}
        \item You can do exactly the same if the base is not ten
        \item Base-10 conversion again by Horner's algorithm \\
        (adapted to the case of negative powers)
    \end{enumerate}
\end{block}
\fi

}

\frame{\frametitle{\ifanswers Horner per i numeri ``frazionari'' \else Horner for ``fractional'' numbers \fi}
\begin{center}
\ifanswers
Chiamiamo (solo in questo corso) frazionari\\ i numeri razionali positivi compresi fra 0 e 1
\else
Let us call (only here) fractional\\ positive rational numbers between 0 and 1
\fi
\end{center}
\begin{columns}
\begin{column}[T]{0.5\textwidth}
\begin{itemize}
\ifanswers
\item Horner per i naturali scorre da sx verso dx
\item moltiplica per la base
\item somma la cifra successiva (a dx)
\else
\item Horner for naturals runs from the left to the right
\item multiply by the base
\item sum the next digit (to the right)
\fi
\end{itemize}
\end{column}
\begin{column}[T]{0.5\textwidth}
\begin{itemize}
\ifanswers
\item Horner per i frazionari scorre da dx verso sx
\item divide per la base
\item somma la cifra successiva (a sx)
\else
\item Horner for fractional numbers runs from the right to the left
\item divide by the base
\item sum the next digit (to the left)
\fi
\end{itemize}
\end{column}
\end{columns}
\begin{center}\small
$.011_2 = 0 \cdot 2^{-1} + 1 \cdot 2^{-2} + 1 \cdot 2^{-3}$
$=\frac{1}{4} + \frac{1}{8} = .375_{10}$
\vskip 3mm
\end{center}
$1/2 + 1 = 3/2$\\
$(3/2)/2 + 0 = 3/4$\\
$(3/4)/2 + 0 = 3/8 = .375$
}

\frame{\frametitle{\ifanswers Horner inverso per numeri frazionari \else Inverse Horner for fractional numbers\fi}
\begin{itemize}
\ifanswers
\item La funzione {\tt mod} \`e il resto della divisione intera
\else
\item Function {\tt mod} gives the reminder of an integer division
\fi
\ifanswers
\item Per frazionari serve il ``resto'' della moltiplicazione intera
\else
\item For fractionals I need the ``reminder'' of an integer multiplication
\fi
\ifanswers
\item Funzione {\tt int} che da' parte intera di un numero\\
	{ \small int(x$\cdot$b) \`e una cifra compresa fra 0 e b-1!}
\else
\item Function {\tt int} gives the integer part of a number\\
{ \small int(x$\cdot$b) is a digit between 0 e b-1!}
\fi
\vskip 2mm
\end{itemize}
int(0.375$\cdot$2)=int(0.75)=0\\
\vskip 1mm 
int((0.75-0)$\cdot$2)=int(1.5)=1\\
\vskip 1mm 
int((1.5-1)$\cdot$2)=int(1)=1\\
\vskip 1mm 
int((1-1)$\cdot$2)=int(0)=0\\
\vskip 1mm 
int((0-0)$\cdot$2)=int(0)=0\\
\vskip 2mm
$0.01100_2$}
\frame{\frametitle{\ifanswers Approssimazione di un numero frazionario \else Approximating a fractional number \fi}
\ifanswers
Approssimare con $n$ cifre un numero frazionario in base $b$?
\else
Approximating with $n$ digits a fractional number in base $b$?
\fi
\begin{itemize}
\ifanswers
\item Troncamento: trascrivo solo le prime $n$ cifre 
\item Arrotondamento: scelgo il numero con $n$ cifre pi\`u vicino
\else
\item Truncation: just writing the first $n$ digits
\item Rounding: take the closest number with $n$ digits
\fi
\end{itemize}
\vskip 2mm
\small
\ifanswers
Approssimare $0.177_{10}$ con un numero di due cifre?
\else
Approximate $0.177_{10}$ as a two-digit number?
\fi
\vskip 1mm
\ifanswers
$0.17$ (troncamento), $0.18$ (arrotondamento)
\else
$0.17$ (truncation), $0.18$ (rounding)
\fi
\vskip 2mm
\ifanswers
Approssimare $0.101_2$ con un numero di due cifre?
\else
Approximate $0.101_2$ as a two-bit number?
\fi
\vskip 1mm
\ifanswers
$0.10_2$ (troncamento), $0.11_2$ (arrotondamento)
\else
$0.10_2$ (truncation), $0.11_2$ (rounding)
\fi
\vskip 3mm
\small
\ifanswers
Troncamento? Banale.\\
Arrotondamento? Guarda solo cifra $n+1$-esima!\\
Base 10? cifra$_{n+1}\geq 5$ eccesso, cifra$_{n+1}<5$ difetto\\
Base 2? cifra$_{n+1}= 1$ eccesso, cifra$_{n+1}=0$ difetto
\else
Truncation? Trivial.\\
Rounding? Consider only $n+1$-th digit!\\
Base 10? digit$_{n+1}\geq 5$ up, digit$_{n+1}<5$ down\\
Base 2? digit$_{n+1}= 1$ up, digit$_{n+1}=0$ down
\fi
}
\end{document}
\frame{\frametitle{Esercizi}
Conversioni:
\begin{itemize}
\item $.5371_{8}=\ldots_{10}=\color{black!20}{.685791015625_{10}}$
\item $.686_{10}=\ldots_{8} =$
	{\tiny$\color{black!20}{.5\overline{3716662132 0 7 1 2 6 0 1 0 1 4 2 2 3 3 5 1 3 6 1 5 2 3 7 5 7 4 7 3 3 1 0 5 5 0 3 4 5 3 0 0 4 0 6 1 1 1 5 6 4 5 7 0 6 5 1 7 6 7 6 3 5 5 4 4 2 6 4 1}}$}
{\tiny$\color{black!20}{\overline{6 2 5 4 0 2 0 3 0 4 4 6 7 2 2 7 4 3 2 4 7 7}_{8}}$}
	\item $.52_{10}=\ldots_2 =\color{black!20}{.\overline{10000101000111101011}_2}$
\item $.1A0F_{16}=\ldots_2=\color{black!20}{.0001101000001111_2}$
\item $.517_8 = \ldots_{16} =\color{black!20}{.101001111_2=.A78_{16}}$
\item $\pi \simeq \ldots_3 = \color{black!20}{10.01020_3}$ (5 cifre dopo virgola arrotodate)
\item ${10.0102_3}=\ldots_{10}$ 
\end{itemize}}


\end{document}

Mi muovo a dx finch\'e 
Nella matrici appare ogni razionale (alcuni pi\`u di una volta es. 1/1 2/2 3/3)

To each point along the spiral, assign a nonnegative integer according 
to its distance from X along this line. Then to every rational number 
in lowest terms there will correspond a positive integer. Every 
rational number is covered, and the correspondence between the 
rational numbers in lowest terms and a certain subset of the integers 
is one-to-one.


-1/1 <--> 0
0/1 <--> 1
1/1 <--> 2
+++++++++++ End of max(a,b) = 1
2/1 <--> 3
(skip 4 because 2/2 is not in lowest terms)
1/2 <--> 5
(skip 6 because 0/2 is not in lowest terms)
-1/2 <--> 7
(skip 8 because -2/2 is not in lowest terms)
-2/1 <--> 9
++++++++++++ End of max(a,b) = 2
-3/1 <--> 10
-3/2 <--> 11
(skip 12 because -3/3 is not in lowest terms)
-2/3 <--> 13
-1/3 <--> 14
(skip 15 because 0/3 is not in lowest terms)
1/3 <--> 16
2/3 <--> 17
(skip 18 because 3/3 is not in lowest terms)
3/2 <--> 19
3/1 <--> 20
++++++++++++ End of max(a,b) = 3
4/1 <--> 21







}

%This implies that the number 
%of integers is no larger than the number of rational numbers.

	\end{document}



\end{document}
\frame{\frametitle{Memorie a $n$ bit}
\begin{columns}
\begin{column}[T]{0.8\textwidth}
\begin{center}
\begin{tikzpicture}
\draw (0,0) +(-.5,-.5) rectangle ++(.5,.5);
\draw (1,0) +(-.5,-.5) rectangle ++(.5,.5);
\draw (3,0) +(-.5,-.5) rectangle ++(.5,.5);
\draw (4,0) +(-.5,-.5) rectangle ++(.5,.5);
\draw (5,0) +(-.5,-.5) rectangle ++(.5,.5);
\draw (2,0) node{$\ldots$};
\draw (0,-.8) node{\tiny bit $n$};
\draw (1,-.8) node{\tiny bit $(n-1)$};
\draw (3,-.8) node{\tiny bit $3$};
\draw (4,-.8) node{\tiny bit $2$};
\draw (5,-.8) node{\tiny bit $1$};
\end{tikzpicture}
\end{center}
\begin{itemize}
\item Singolo bit ({\bf b}inary dig{\bf it})  assume solo valori $0$ o $1$
\item Memoria a $n$ bit? $2^n$ configurazioni
\item Rappresentare i naturali? Sistema posizionale\\
Con $n$ bit, rappresento range $\{ 0 , 1 , \ldots , 2^{n}-1 \} \subset \mathbb{N}$
\item Rappresentare gli interi? Idee?
\item Un bit per il segno, il resto posizionale?
\item Non compatto e no somma in colonna
\item Metodo alternativo: complemento a due!
\end{itemize}
\end{column}
\begin{column}[T]{0.3\textwidth}
\setbeamercovered{}
\begin{tikzpicture}
\draw (0,0) +(-.2,-.2) rectangle ++(.2,.2); \draw (0,0) node{\tiny $1$};
\draw (.4,0) +(-.2,-.2) rectangle ++(.2,.2); \draw (.4,0) node{\tiny $1$};
\draw (.8,0) +(-.2,-.2) rectangle ++(.2,.2); \draw (.8,0) node{\tiny $1$};
\draw (1.2,0) +(-.2,-.2) rectangle ++(.2,.2); \draw (1.2,0) node{\tiny $1$};
\draw (0,.5) +(-.2,-.2) rectangle ++(.2,.2); \draw (0,.5) node{\tiny $1$};
\draw (.4,.5) +(-.2,-.2) rectangle ++(.2,.2); \draw (.4,.5) node{\tiny $1$};
\draw (.8,.5) +(-.2,-.2) rectangle ++(.2,.2); \draw (.8,.5) node{\tiny $1$};
\draw (1.2,.5) +(-.2,-.2) rectangle ++(.2,.2); \draw (1.2,.5) node{\tiny $0$};
\draw (0,1) +(-.2,-.2) rectangle ++(.2,.2); \draw (0,1) node{\tiny $1$};
\draw (.4,1) +(-.2,-.2) rectangle ++(.2,.2); \draw (.4,1) node{\tiny $1$};
\draw (.8,1) +(-.2,-.2) rectangle ++(.2,.2); \draw (.8,1) node{\tiny $0$};
\draw (1.2,1) +(-.2,-.2) rectangle ++(.2,.2); \draw (1.2,1) node{\tiny $1$};
\draw (0,1.5) +(-.2,-.2) rectangle ++(.2,.2); \draw (0,1.5) node{\tiny $1$};
\draw (.4,1.5) +(-.2,-.2) rectangle ++(.2,.2); \draw (.4,1.5) node{\tiny $1$};
\draw (.8,1.5) +(-.2,-.2) rectangle ++(.2,.2); \draw (.8,1.5) node{\tiny $0$};
\draw (1.2,1.5) +(-.2,-.2) rectangle ++(.2,.2); \draw (1.2,1.5) node{\tiny $0$};
\draw (0,2) +(-.2,-.2) rectangle ++(.2,.2); \draw (0,2) node{\tiny $1$};
\draw (.4,2) +(-.2,-.2) rectangle ++(.2,.2); \draw (.4,2) node{\tiny $0$};
\draw (.8,2) +(-.2,-.2) rectangle ++(.2,.2); \draw (.8,2) node{\tiny $1$};
\draw (1.2,2) +(-.2,-.2) rectangle ++(.2,.2); \draw (1.2,2) node{\tiny $1$};
\draw (0,2.5) +(-.2,-.2) rectangle ++(.2,.2); \draw (0,2.5) node{\tiny $1$};
\draw (.4,2.5) +(-.2,-.2) rectangle ++(.2,.2); \draw (.4,2.5) node{\tiny $0$};
\draw (.8,2.5) +(-.2,-.2) rectangle ++(.2,.2); \draw (.8,2.5) node{\tiny $1$};
\draw (1.2,2.5) +(-.2,-.2) rectangle ++(.2,.2); \draw (1.2,2.5) node{\tiny $0$};
\draw (0,3) +(-.2,-.2) rectangle ++(.2,.2); \draw (0,3) node{\tiny $1$};
\draw (.4,3) +(-.2,-.2) rectangle ++(.2,.2); \draw (.4,3) node{\tiny $0$};
\draw (.8,3) +(-.2,-.2) rectangle ++(.2,.2); \draw (.8,3) node{\tiny $0$};
\draw (1.2,3) +(-.2,-.2) rectangle ++(.2,.2); \draw (1.2,3) node{\tiny $1$};
\draw (0,3.5) +(-.2,-.2) rectangle ++(.2,.2); \draw (0,3.5) node{\tiny $1$};
\draw (.4,3.5) +(-.2,-.2) rectangle ++(.2,.2); \draw (.4,3.5) node{\tiny $0$};
\draw (.8,3.5) +(-.2,-.2) rectangle ++(.2,.2); \draw (.8,3.5) node{\tiny $0$};
\draw (1.2,3.5) +(-.2,-.2) rectangle ++(.2,.2); \draw (1.2,3.5) node{\tiny $0$};
\draw (0,4) +(-.2,-.2) rectangle ++(.2,.2); \draw (0,4) node{\tiny $0$};
\draw (.4,4) +(-.2,-.2) rectangle ++(.2,.2); \draw (.4,4) node{\tiny $1$};
\draw (.8,4) +(-.2,-.2) rectangle ++(.2,.2); \draw (.8,4) node{\tiny $1$};
\draw (1.2,4) +(-.2,-.2) rectangle ++(.2,.2); \draw (1.2,4) node{\tiny $1$};
\draw (0,4.5) +(-.2,-.2) rectangle ++(.2,.2); \draw (0,4.5) node{\tiny $0$};
\draw (.4,4.5) +(-.2,-.2) rectangle ++(.2,.2); \draw (.4,4.5) node{\tiny $1$};
\draw (.8,4.5) +(-.2,-.2) rectangle ++(.2,.2); \draw (.8,4.5) node{\tiny $1$};
\draw (1.2,4.5) +(-.2,-.2) rectangle ++(.2,.2); \draw (1.2,4.5) node{\tiny $0$};
\draw (0,5) +(-.2,-.2) rectangle ++(.2,.2); \draw (0,5) node{\tiny $0$};
\draw (.4,5) +(-.2,-.2) rectangle ++(.2,.2); \draw (.4,5) node{\tiny $1$};
\draw (.8,5) +(-.2,-.2) rectangle ++(.2,.2); \draw (.8,5) node{\tiny $0$};
\draw (1.2,5) +(-.2,-.2) rectangle ++(.2,.2); \draw (1.2,5) node{\tiny $1$};
\draw (0,5.5) +(-.2,-.2) rectangle ++(.2,.2); \draw (0,5.5) node{\tiny $0$};
\draw (.4,5.5) +(-.2,-.2) rectangle ++(.2,.2); \draw (.4,5.5) node{\tiny $1$};
\draw (.8,5.5) +(-.2,-.2) rectangle ++(.2,.2); \draw (.8,5.5) node{\tiny $0$};
\draw (1.2,5.5) +(-.2,-.2) rectangle ++(.2,.2); \draw (1.2,5.5) node{\tiny $0$};
\draw (0,6) +(-.2,-.2) rectangle ++(.2,.2); \draw (0,6) node{\tiny $0$};
\draw (.4,6) +(-.2,-.2) rectangle ++(.2,.2); \draw (.4,6) node{\tiny $0$};
\draw (.8,6) +(-.2,-.2) rectangle ++(.2,.2); \draw (.8,6) node{\tiny $1$};
\draw (1.2,6) +(-.2,-.2) rectangle ++(.2,.2); \draw (1.2,6) node{\tiny $1$};
\draw (0,6.5) +(-.2,-.2) rectangle ++(.2,.2); \draw (0,6.5) node{\tiny $0$};
\draw (.4,6.5) +(-.2,-.2) rectangle ++(.2,.2); \draw (.4,6.5) node{\tiny $0$};
\draw (.8,6.5) +(-.2,-.2) rectangle ++(.2,.2); \draw (.8,6.5) node{\tiny $1$};
\draw (1.2,6.5) +(-.2,-.2) rectangle ++(.2,.2); \draw (1.2,6.5) node{\tiny $0$};
\draw (0,7) +(-.2,-.2) rectangle ++(.2,.2); \draw (0,7) node{\tiny $0$};
\draw (.4,7) +(-.2,-.2) rectangle ++(.2,.2); \draw (.4,7) node{\tiny $0$};
\draw (.8,7) +(-.2,-.2) rectangle ++(.2,.2); \draw (.8,7) node{\tiny $0$};
\draw (1.2,7) +(-.2,-.2) rectangle ++(.2,.2); \draw (1.2,7) node{\tiny $1$};
\draw (0,7.5) +(-.2,-.2) rectangle ++(.2,.2); \draw (0,7.5) node{\tiny $0$};
\draw (.4,7.5) +(-.2,-.2) rectangle ++(.2,.2); \draw (.4,7.5) node{\tiny $0$};
\draw (.8,7.5) +(-.2,-.2) rectangle ++(.2,.2); \draw (.8,7.5) node{\tiny $0$};
\draw (1.2,7.5) +(-.2,-.2) rectangle ++(.2,.2); \draw (1.2,7.5) node{\tiny $0$};

\draw (-.7,0) node{\tiny $15=$}; \draw (-.7,.5) node{\tiny $14=$}; \draw (-.7,1) node{\tiny $13=$}; \draw (-.7,1.5) node{\tiny $12=$}; \draw (-.7,2) node{\tiny $11=$}; \draw (-.7,2.5) node{\tiny $10=$}; \draw (-.7,3) node{\tiny $9=$}; \draw (-.7,3.5) node{\tiny $8=$}; \draw (-.7,4) node{\tiny $7=$}; \draw (-.7,4.5) node{\tiny $6=$}; \draw (-.7,5) node{\tiny $5=$}; \draw (-.7,5.5) node{\tiny $4=$}; \draw (-.7,6) node{\tiny $3=$}; \draw (-.7,6.5) node{\tiny $2=$}; \draw (-.7,7) node{\tiny $1=$}; \draw (-.7,7.5) node{\tiny $= 0$};
\draw (1.7,0) node{\tiny $= -1$}; \draw (1.7,.5) node{\tiny $= -2$}; \draw (1.7,1) node{\tiny $= -3$}; \draw (1.7,1.5) node{\tiny $= -4$}; \draw (1.7,2) node{\tiny $= -5$};
\draw (1.7,2.5) node{\tiny $= -6$}; \draw (1.7,3) node{\tiny $= -7$}; \draw (1.7,3.5) node{\tiny $= -8$}; \draw (1.7,4) node{\tiny $= 7$}; \draw (1.7,4.5) node{\tiny $= 6$};
\draw (1.7,5) node{\tiny $= 5$}; \draw (1.7,5.5) node{\tiny $= 4$}; \draw (1.7,6) node{\tiny $= 3$}; \draw (1.7,6.5) node{\tiny $= 2$}; \draw (1.7,7) node{\tiny $= 1$}; \draw (1.7,7.5) node{\tiny $= 0$};

\end{tikzpicture}
\end{column}
\end{columns}}



\end{document}
